% Preamble.
\documentclass[12pt]{article}
\usepackage[margin=1.25in]{geometry}
\usepackage[fleqn]{amsmath}
\usepackage{textcomp}
\usepackage{gensymb}
\usepackage{amsfonts}
\usepackage{enumitem}
\usepackage{siunitx}
%\usepackage{subfiles}  % Include for subfiles.
%\usepackage{tikz}  % Include for handmade figures.
%\usepackage{graphicx}  % Include for external figures.

%% Title macros.
\newcommand{\HOMEWORKNUM}{3}
\newcommand{\NAME}{D. Choi}
\newcommand{\DATE}{2020-09-11}

\title{\vspace{-2\baselineskip}PHYS 151 - Written Homework \#\HOMEWORKNUM}
\author{\NAME}
\date{\DATE}

%% Formatting options.
%\pagenumbering{gobble}  % Include for single-page document.


% Document.
\begin{document}
\maketitle


\section*{1.}
\textit{The figure shows a sailing ship approaching a hostile harbor. At time
$t = 0$, a cannon placed atop a lighthouse, at height $h$ above the deck of
the hostile ship, fires a projectile at the oncoming ship, which is moving at
a speed of $v_1$ toward the shore. The cannonball is fired at a speed of $v_2$
at an angle $\theta$ above the horizontal.}
\begin{enumerate}[label=(\alph*)]
	\item \textit{For this cannonball to impact exactly the front of the ship,
	at what distance $D$ from the dock should the ship be when the equipment
	is thrown? Ignore air resistance.} \\[\baselineskip]
	The ship's horizontal and vertical positions as functions of time can be
	modeled by
	\begin{gather*}
		x_s(t) = -v_1t + D, \\
		y_s(t) = 0.
	\end{gather*}
	In turn, the projectile's horizontal and vertical positions can be modeled
	by
	\begin{gather*}
		x_p(t) = v_2\cos(\theta)t, \\
		y_p(t) = -\frac{g}{2}t^2 + v_2\sin(\theta)t + h.
	\end{gather*}
	The projectile must hit the ship at a time $t_f$ when both
	$x_s(t_f) = x_p(t_f)$ and $y_s(t_f) = y_p(t_f)$. Because $y_s(t_f)$ and
	$y_p(t_f)$ are both known, $t_f$ can be solved for with
	\begin{gather*}
		y_s(t_f) = y_p(t_f), \\
		0 = -\frac{g}{2}{t_f}^2 + v_2\sin(\theta)t_f + h, \\
		t_f = \frac{v_2\sin(\theta) + \sqrt{(v_2\sin(\theta))^2 + 2gh}}{g}.
	\end{gather*}
	Using $t_f$, $D$ can be solved for with
	\begin{gather*}
		x_s(t_f) = x_p(t_f), \\
		-v_1t_f + D = v_2\cos(\theta)t_f, \\
		D = v_2\cos(\theta)t_f + v_1t_f, \\
		D = (v_2\cos(\theta) + v_1) t_f.
	\end{gather*}
	In summary, the expanded formula to find the distance $D$ is
	\begin{gather*}
		\boxed{D = (v_2\cos(\theta) + v_1) t_f}
	\end{gather*}
	where
	\begin{equation*}
		\boxed{
			t_f = \frac{v_2\sin(\theta)
			+ \sqrt{(v_2\sin(\theta))^2 + 2gh}}{g}
		}.
	\end{equation*}
	\item \textit{What is the total time of flight of the cannonball? \\
	Use $v_1 = 40.0\si{\frac{cm}{s}},\ v_2 = 16.0\si{\frac{m}{s}},\ 
	\theta = \ang{55.0}$, and $h = 9.25\si{m}$.}
	\begin{align*}
		t_f
		&= \frac{v_2\sin(\theta) + \sqrt{(v_2\sin(\theta))^2 + 2gh}}{g} \\
		&= \frac
			{
				(16.0\si{\frac{m}{s}}) \sin(\ang{55.0})
				+ \sqrt{
					((16.0\si{\frac{m}{s}}) \sin(\ang{55.0}))^2
					+ 2 (9.81\si{\frac{m}{s^2}}) (9.25\si{m})
				}
			}
			{9.81\si{\frac{m}{s^2}}}
			\\
		&= \boxed{3.25\si{s}}.
	\end{align*}
	\item \textit{Using the numbers given in (b.), what are the $x$ and $y$
	components of the velocity of the cannonball just before impacting the
	deck of the ship?} \\[\baselineskip]
	Let $v_x(t)$ and $v_y(t)$ represent the horizontal and vertical components
	of the projectile's velocity as functions of time, respectively.
	\begin{align*}
		v_x(t_f)
		&= v_2\cos(\theta) \\
		&= (16.0\si{\frac{m}{s}}) \cos(\ang{55.0}) \\
		&= \boxed{9.18\si{\frac{m}{s}}}.
	\end{align*}
	\begin{align*}
		v_y(t_f)
		&= -gt_f + v_2\sin(\theta) \\
		&= -(9.81\si{\frac{m}{s^2}}) (3.25\si{s})
			+ (16.0\si{\frac{m}{s}}) \sin(\ang{55.0}) \\
		&= \boxed{-18.8\si{\frac{m}{s}}}.
	\end{align*}
\end{enumerate}


\end{document}
