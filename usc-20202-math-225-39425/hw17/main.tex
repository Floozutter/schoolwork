% Preamble.
\documentclass[12pt]{article}
\usepackage[margin=1.25in]{geometry}
\usepackage[fleqn]{amsmath}
\usepackage{textcomp}
\usepackage{gensymb}
\usepackage{amsfonts}
\usepackage{enumitem}
%\usepackage{tikz}  % Include for figures.
%\usepackage{subfiles}  % Include for subfiles.

%% Title macros.
\newcommand{\HOMEWORKNUM}{17}
\newcommand{\NAME}{D. Choi}
\newcommand{\DATE}{2020-06-18}

\title{\vspace{-2\baselineskip}MATH 225 - Homework \#\HOMEWORKNUM}
\author{\NAME}
\date{\DATE}

%% Formatting options.
%\pagenumbering{gobble}  % Include for single-page document.


% Document.
\begin{document}
\maketitle

\section*{1.}
\textit{Find the matrix that rotates space $30 \degree$ counterclockwise about
the vector $\vec{n} = \begin{pmatrix} 1 \\ 0 \\ -1 \end{pmatrix}$.}
\\[\baselineskip]
The plane normal to $\vec{n}$ can be represented by the equation
\begin{equation*}
	0
	=
	\vec{n} \cdot \begin{pmatrix} x \\ y \\ z \end{pmatrix}
	=
	\begin{pmatrix} 1 \\ 0 \\ -1 \end{pmatrix}
	\cdot
	\begin{pmatrix} x \\ y \\ z \end{pmatrix}
	=
	x - z
	.
\end{equation*}
Let $\vec{v}_1 = \vec{n}$. \\
A vector that is orthogonal to $\vec{v}_1$ can be found on the plane
$0 = x - z$:
\begin{equation*}
	\vec{v}_2 = \begin{pmatrix} 1 \\ 0 \\ 1 \end{pmatrix}
	.
\end{equation*}
The vector that is orthogonal to both $\vec{v}_1$ and $\vec{v}_2$ can be found
by computing the cross product of $\vec{v}_1$ and $\vec{v}_2$:
\begin{equation*}
	\vec{v}_3 = \vec{v}_1 \times \vec{v}_2 =
	\begin{pmatrix} 0 \\ -2 \\ 0 \end{pmatrix}
	.
\end{equation*}
A convenient basis for describing the rotation can be constructed from
normalizing the orthogonal triple $\vec{v}_1$, $\vec{v}_2$, and $\vec{v}_3$.
Let such a basis be $\mathcal{B}$, where
\begin{equation*}
	\mathcal{B}
	=
	\left\{ \hat{v}_1, \hat{v}_2, \hat{v}_3 \right\}
	=
	\left\{
		\begin{pmatrix}
			\frac{1}{\sqrt{2}} \\
			0 \\
			-\frac{1}{\sqrt{2}}
		\end{pmatrix}
		,
		\begin{pmatrix}
			\frac{1}{\sqrt{2}} \\
			0 \\
			\frac{1}{\sqrt{2}}
		\end{pmatrix}
		,
		\begin{pmatrix}
			0 \\
			-1 \\
			0
		\end{pmatrix}
	\right\}
	,
\end{equation*}
in contrast to the standard basis $\mathcal{S}$, where
\begin{equation*}
	\mathcal{S}
	=
	\left\{
		\begin{pmatrix} 1 \\ 0 \\ 0 \end{pmatrix},
		\begin{pmatrix} 0 \\ 1 \\ 0 \end{pmatrix},
		\begin{pmatrix} 0 \\ 0 \\ 1 \end{pmatrix}
	\right\}
	.
\end{equation*}
\newpage \noindent
In basis $\mathcal{B}$ the rotation of interest is a $30 \degree$
counterclockwise rotation about $\hat{v}_1$. \\
Let ${\{R\}}_{\mathcal{B}}^{\mathcal{B}}$ be the matrix for such a rotation in
basis $\mathcal{B}$ so that
\begin{equation*}
	{\{R\}}_{\mathcal{B}}^{\mathcal{B}}
	=
	\begin{pmatrix}
		1 & 0 & 0 \\
		0 & \cos(30 \degree) & -\sin(30 \degree) \\
		0 & \sin(30 \degree) & \cos(30 \degree)
	\end{pmatrix}
	.
\end{equation*}
Let ${\{I\}}_{\mathcal{B}}^{\mathcal{S}}$ be the matrix with columns
$\hat{v}_1$, $\hat{v}_2$, and $\hat{v}_3$ so that
\begin{equation*}
	{\{I\}}_{\mathcal{B}}^{\mathcal{S}}
	=
	\begin{pmatrix}
		\frac{1}{\sqrt{2}} & \frac{1}{\sqrt{2}} & 0 \\
		0 & 0 & -1 \\
		-\frac{1}{\sqrt{2}} & \frac{1}{\sqrt{2}} & 0
	\end{pmatrix}
	.
\end{equation*}
${\{I\}}_{\mathcal{B}}^{\mathcal{S}}$ can be thought of as the matrix that
converts an input ${\{\vec{x}\}}_{\mathcal{B}}$ (a vector described in basis
$\mathcal{B}$) to ${\{\vec{x}\}}_{\mathcal{S}}$ (the same vector but
described in basis $\mathcal{S}$). \\
Similarly, let ${\{I\}}_{\mathcal{S}}^{\mathcal{B}}$ be the matrix that
converts from basis $\mathcal{S}$ to $\mathcal{B}$, so that
\begin{equation*}
	{\{I\}}_{\mathcal{S}}^{\mathcal{B}}
	=
	{\left( {\{I\}}_{\mathcal{B}}^{\mathcal{S}} \right)}^{-1}
	=
	{\left( {\{I\}}_{\mathcal{B}}^{\mathcal{S}} \right)}^{\text{T}}
	=
	\begin{pmatrix}
		\frac{1}{\sqrt{2}} & 0 & -\frac{1}{\sqrt{2}} \\
		\frac{1}{\sqrt{2}} & 0 & \frac{1}{\sqrt{2}} \\
		0 & -1 & 0
	\end{pmatrix}
	.
\end{equation*}
Let ${\{R\}}_{\mathcal{S}}^{\mathcal{S}}$ be the rotation of interest in
standard basis. \\
${\{R\}}_{\mathcal{S}}^{\mathcal{S}}$ can be computed as the composition of
firstly the transformation from basis $\mathcal{S}$ to $\mathcal{B}$, then the
rotation of interest under basis $\mathcal{B}$, and finally the transformation
from basis $\mathcal{B}$ back to $\mathcal{S}$:
\begin{align*}
	{\{R\}}_{\mathcal{S}}^{\mathcal{S}}
	&=
	{{\{I\}}_{\mathcal{B}}^{\mathcal{S}}}
	\cdot
	{{\{R\}}_{\mathcal{B}}^{\mathcal{B}}}
	\cdot
	{{\{I\}}_{\mathcal{S}}^{\mathcal{B}}}
	\\
	&=
	\boxed{
		\begin{pmatrix}
			\frac{1}{\sqrt{2}} & \frac{1}{\sqrt{2}} & 0 \\
			0 & 0 & -1 \\
			-\frac{1}{\sqrt{2}} & \frac{1}{\sqrt{2}} & 0
		\end{pmatrix}
		\begin{pmatrix}
			1 & 0 & 0 \\
			0 & \cos(30 \degree) & -\sin(30 \degree) \\
			0 & \sin(30 \degree) & \cos(30 \degree)
		\end{pmatrix}
		\begin{pmatrix}
			\frac{1}{\sqrt{2}} & 0 & -\frac{1}{\sqrt{2}} \\
			\frac{1}{\sqrt{2}} & 0 & \frac{1}{\sqrt{2}} \\
			0 & -1 & 0
		\end{pmatrix}
	}
	.
\end{align*}

\section*{2.}
\textit{Let $\vec{a} = \begin{pmatrix} 5 \\ 0 \\ 0 \end{pmatrix}$ and
$\vec{b} = \begin{pmatrix} 0 \\ 3 \\ 4 \end{pmatrix}$. \\
Find the vector $\vec{v}$ that is $\vec{a}$ rotated $40 \degree$ toward
$\vec{b}$.} \\[\baselineskip]
Note that $\vec{a} \cdot \vec{b} = 0$, meaning that the two vectors are
orthogonal. \\
A vector $\vec{c}$ in the axis of rotation can be found by computing
\begin{equation*}
	\vec{c}
	=
	\vec{a} \times \vec{b}
	=
	\begin{pmatrix} 0 \\ -20 \\ 15 \end{pmatrix}
	.
\end{equation*}
Let the basis $\mathcal{B}$ be constructed from the triple $\vec{a}$,
$\vec{b}$, and $\vec{c}$:
\begin{equation*}
	\mathcal{B}
	=
	\left\{ \vec{a}, \vec{b}, \vec{c} \right\}
	=
	\left\{
		\begin{pmatrix} 5 \\ 0 \\ 0 \end{pmatrix},
		\begin{pmatrix} 0 \\ 3 \\ 4 \end{pmatrix},
		\begin{pmatrix} 0 \\ -20 \\ 15 \end{pmatrix}
	\right\}
	.
\end{equation*}
Let ${\{R\}}_{\mathcal{B}}^{\mathcal{B}}$ be the matrix for the rotation of
interest about $\vec{c}$ in basis $\mathcal{B}$:
\begin{equation*}
	{\{R\}}_{\mathcal{B}}^{\mathcal{B}}
	=
	\begin{pmatrix}
		\cos(40 \degree) & -\sin(40 \degree) & 0 \\
		\sin(40 \degree) & \cos(40 \degree) & 0 \\
		0 & 0 & 1
	\end{pmatrix}
	.
\end{equation*}
Using ${\{R\}}_{\mathcal{B}}^{\mathcal{B}}$, 
\begin{gather*}
	{\{\vec{v}\}}_{\mathcal{B}}
	=
	{\{R\}}_{\mathcal{B}}^{\mathcal{B}}
	{\{\vec{a}\}}_{\mathcal{B}}
	=
	{\{R\}}_{\mathcal{B}}^{\mathcal{B}}
	\begin{pmatrix} 1 \\ 0 \\ 0 \end{pmatrix}
	=
	\begin{pmatrix}
		\cos(40 \degree) \\
		\sin(40 \degree) \\
		0
	\end{pmatrix}
	,
	\\
	\vec{v}
	=
	\cos(40 \degree)
	\begin{pmatrix} 5 \\ 0 \\ 0 \end{pmatrix}
	+
	\sin(40 \degree)
	\begin{pmatrix} 0 \\ 3 \\ 4 \end{pmatrix}
	+
	0
	\begin{pmatrix} 0 \\ -20 \\ 15 \end{pmatrix}
	=
	\boxed{
		\begin{pmatrix}
			5 \cos(40 \degree) \\
			3 \sin(40 \degree) \\
			4 \sin(40 \degree)
		\end{pmatrix}
	}
	.
\end{gather*}

\end{document}