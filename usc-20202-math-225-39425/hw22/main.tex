% Preamble.
\documentclass[12pt]{article}
\usepackage[margin=1.25in]{geometry}
\usepackage[fleqn]{amsmath}
\usepackage{textcomp}
\usepackage{gensymb}
\usepackage{amsfonts}
\usepackage{enumitem}
%\usepackage{tikz}  % Include for figures.
%\usepackage{subfiles}  % Include for subfiles.

%% Title macros.
\newcommand{\HOMEWORKNUM}{22}
\newcommand{\NAME}{D. Choi}
\newcommand{\DATE}{2020-06-25}

\title{\vspace{-2\baselineskip}MATH 225 - Homework \#\HOMEWORKNUM}
\author{\NAME}
\date{\DATE}

%% Formatting options.
%\pagenumbering{gobble}  % Include for single-page document.


% Document.
\begin{document}
\maketitle

\section*{1.}
\textit{For the following matrices, find the eigenvectors and associated
eigenvalues by thinking geometrically about the corresponding matrix
transformation.} \\[\baselineskip]
\begin{enumerate}[label=(\alph*)]
	\item $\begin{pmatrix} 3 & 0 \\ 0 & 3 \end{pmatrix}$. \\[\baselineskip]
	This matrix transformation is equivalent to scaling a vector by $3$:
	\begin{equation*}
		\begin{pmatrix} 3 & 0 \\ 0 & 3 \end{pmatrix}
		\vec{v}
		=
		3 \vec{v}
		.
	\end{equation*}
	Thus, every nonzero vector is an eigenvector of this matrix. Their
	corresponding eigenvalue is \boxed{3}.
	
	\item $\begin{pmatrix} -2 & 0 \\ 0 & 4 \end{pmatrix}$. \\[\baselineskip]
	This matrix transformation is equivalent to scaling a vector's first
	component by $-2$ and its second component by $4$.
	\begin{equation*}
		\begin{pmatrix} -2 & 0 \\ 0 & 4 \end{pmatrix}
		\begin{pmatrix} x \\ y \end{pmatrix}
		=
		\begin{pmatrix} -2x \\ 4y \end{pmatrix}
		.
	\end{equation*}
	Thus, nonzero vectors in the form of
	$\begin{pmatrix} n \\ 0 \end{pmatrix}$ are eigenvectors of this matrix,
	with a corresponding eigenvalue of \boxed{-2}, and nonzero vectors in the
	form of $\begin{pmatrix} 0 \\ n \end{pmatrix}$ are eigenvectors of this
	matrix, with a corresponding eigenvalue of \boxed{4}.
	
	\item \textit{The identity matrix.} \\[\baselineskip]
	The identity transformation is equivalent to scaling a vector by $1$:
	\begin{equation*}
		I \vec{v} = 1 \vec{v}.
	\end{equation*}
	Thus, every nonzero vector is an eigenvector of the identity matrix. Their
	corresponding eigenvalue is \boxed{1}.
	
	\item \textit{A diagonal matrix with distinct diagonal entries.}
	\\[\baselineskip]
	This matrix transformation is equivalent to scaling a vector's entries
	each by their corresponding diagonal entry. \\
	Thus, such a matrix's eigenvectors are scalar multiples of the standard
	basis vectors. Each eigenvector's corresponding diagonal entry is their
	corresponding eigenvalue.
\end{enumerate}

\section*{2.}
\textit{Suppose that $A$ is a $2 \times 2$ matrix having eigenvectors}
\begin{equation*}
	\vec{v}_1 = \begin{pmatrix} 2 \\ 1 \end{pmatrix}, \quad
	\vec{v}_2 = \begin{pmatrix} -1 \\ 2 \end{pmatrix}
\end{equation*}
\textit{and associated eigenvalues $\lambda_1 = 2$, $\lambda_2 = -3$. If
$\vec{x} = \begin{pmatrix} 5 \\ 0 \end{pmatrix}$, find the vector
$A^4 \vec{x}$.}

\section*{3.}
\textit{Determine whether the following statements are true or false and
provide a justification for your response.}
\begin{enumerate}[label=(\alph*)]
	\item \textit{The eigenvalues of a diagonal matrix are equal to the
	entries on the diagonal.}
	\item \textit{If $A v = \lambda v$, then $A^2 v = \lambda v$ as well.}
	\item \textit{Every vector is an eigenvector of the identity matrix.}
	\item \textit{If $\lambda = 0$ is an eigenvalue of $A$, then $A$ is
	invertible.}
	\item \textit{For every $n \times n$ matrix $A$, it is possible to find
	a basis of $\mathbb{R}^n$ consisting of eigenvectors of $A$.}
\end{enumerate}

\end{document}