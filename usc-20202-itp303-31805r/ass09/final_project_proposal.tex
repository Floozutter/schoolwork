% Preamble.
\documentclass[12pt]{article}
\usepackage{hyperref}

%% Title
\title{\vspace{-4\baselineskip}ITP 303 - Assignment \#9 \\
Final Project Proposal}
\author{D. Choi}
\date{2020-06-26}

% Document.
\begin{document}
\maketitle

\section*{Topic}
The website, Langalike, will be centered around a quiz that challenges
visitors to choose which programming language names are real out of a pool of
fake names. The website's quiz can be used as a sort of weak litmus test for
how acquainted a person is with different areas of programming. Visitors who
have completed the quiz will be able to submit their scores under a username.

\section*{Audience}
The intended audience for the website is anyone exposed to the topic of
computer programming with varying degrees of familiarity. Different types of
programming languages will be explored in the quiz so that it will be
compatible for a broad range of programmers.

\section*{Design and Style}
The design and color scheme will be influenced by code-adjacent environments
such as the command-line terminals or code editors. A monospaced font will be
used. Colors will take inspiration from common code syntax highlighting
tropes. \\[\baselineskip]
Three websites with similar design philosophies:
\begin{enumerate}
	\item Google Foobar - \url{https://foobar.withgoogle.com/}
	\item Advent of Code - \url{https://adventofcode.com/}
	\item christine.website - \url{https://christine.website/}
\end{enumerate}
\newpage

\section*{Scope}
The website will have 4 separate pages, a page for the quiz, an about page,
a user profile page where the user can submit their quiz score and leave a
comment, and a global score page where the user can see other users' scores.

\section*{Database}
User quiz scores and comments will be stored in the database.

\end{document}