% Preamble.
\documentclass[12pt]{article}
\usepackage[margin=1.25in]{geometry}
\usepackage[fleqn]{amsmath}
\usepackage{textcomp}
\usepackage{gensymb}
\usepackage{amsfonts}
\usepackage{enumitem}
\usepackage{siunitx}
%\usepackage{subfiles}  % Include for subfiles.
%\usepackage{tikz}  % Include for handmade figures.
\usepackage{graphicx}  % Include for external figures.

%% Title macros.
\newcommand{\HOMEWORKNUM}{1}
\newcommand{\NAME}{D. Choi}
\newcommand{\DATE}{2020-08-28}

\title{\vspace{-2\baselineskip}PHYS 225 - Written Homework \#\HOMEWORKNUM}
\author{\NAME}
\date{\DATE}

%% Formatting options.
%\pagenumbering{gobble}  % Include for single-page document.


% Document.
\begin{document}
\maketitle


\section*{1.}
\textit{Consider the vectors $\vec{A} = -2\hat{\imath} - 2\hat{\jmath}$ and
$\vec{B} = \hat{\imath} + 3\hat{\jmath}$.}
\begin{enumerate}[label=(\alph*)]
	\item \textit{Find the unit vectors
	$\hat{A} = \frac{\vec{A}}{\|\vec{A}\|}$ and
	$\hat{B} = \frac{\vec{B}}{\|\vec{B}\|}$.}
	\begin{gather*}
		\hat{A}
		= \frac{\vec{A}}{\|\vec{A}\|}
		= \frac{-2\hat{\imath} - 2\hat{\jmath}}{2\sqrt{2}}
		= \boxed{
			-\frac{1}{\sqrt{2}} \hat{\imath}
			- \frac{1}{\sqrt{2}} \hat{\jmath}
		}, \\
		\hat{B}
		= \frac{\vec{B}}{\|\vec{B}\|}
		= \frac{\hat{\imath} + 3\hat{\jmath}}{\sqrt{10}}
		= \boxed{
			\frac{1}{\sqrt{10}} \hat{\imath}
			+ \frac{3}{\sqrt{10}} \hat{\jmath}
		}.
	\end{gather*}
	\item \textit{Express the vector $\vec{C} = 5\hat{\imath} - 3\hat{\jmath}$
	in terms of $\hat{A}$ and $\hat{B}$.}
	\begin{gather*}
		\vec{C} = \hat{A}x + \hat{B}y, \\
		\begin{pmatrix} 5 \\ -3 \end{pmatrix}
		=
		\begin{pmatrix}
			-\frac{1}{\sqrt{2}} & \frac{1}{\sqrt{10}} \\
			-\frac{1}{\sqrt{2}} & \frac{3}{\sqrt{10}}
		\end{pmatrix}
		\begin{pmatrix} x \\ y \end{pmatrix}.
	\end{gather*}
	Solving for $x$ and $y$,
	\begin{equation*}
		\begin{pmatrix} x \\ y \end{pmatrix}
		= \begin{pmatrix} -9\sqrt{2} \\ -4\sqrt{10} \end{pmatrix}.
	\end{equation*}.
	Thus,
	\begin{equation*}
		\vec{C} = \boxed{
			-9\sqrt{2}\hat{A} - 4\sqrt{10}\hat{B}
		}.
	\end{equation*}
\end{enumerate}

\section*{2.}
\textit{A particle moves in one dimension with a velocity
$v(t) = v_0 + v_1 t^3$, where $v_0 = 2.00\si{\frac{m}{s}}$ and
$v_1 = 1.00\si{\frac{m}{s^4}}$.}
\begin{enumerate}[label=(\alph*)]
	\item \textit{Assuming that at $t_0 = 0$, the particle is at position
	$x_0 = 4.00\si{m}$ measured from the origin, derive a formula for the
	particle's position as a function of time.}
	\begin{gather*}
		v(t) = v_1 t^3 + v_0, \\
		x(t)
		= \int v(t) dt
		= \boxed{
			\frac{v_1}{4}t^4 + v_0 t + x_0
		}.
	\end{gather*}
	\item \textit{Obtain a formula for the acceleration of the particle.}
	\begin{equation*}
		a(t)
		= \frac{d}{dt} v(t)
		= \boxed{
			3v_1 t^2
		}.
	\end{equation*}
	\item \textit{Make quantitative plots of the particle's position,
	velocity, and acceleration as a function of time during the interval
	$t_0 = 0$ and $t_1 = 1.00\si{s}$.}
	\begin{center}
		\includegraphics[width=0.6\textwidth]{fig.pdf}
	\end{center}
	\newpage
	\item \textit{What is the particle's average velocity and acceleration in
	the time interval $t_0 = 0$ to $t_1 = 1.00\si{s}$?} \\[\baselineskip]
	For the interval $[0, 1.00\si{s}]$, let $v_{avg}$ and $a_{avg}$ be the
	particle's average velocity and average acceleration over the interval
	respectively.
	\begin{gather*}
		v_{avg}
		= \frac{x(1.00\si{s}) - x(0)}{1.00\si{s} - 0}
		= \frac{6.25\si{m} - 4.00\si{m}}{1.00\si{s} - 0}
		= \boxed{2.25\si{\frac{m}{s}}}, \\
		a_{avg}
		= \frac{v(1.00\si{s}) - v(0)}{1.00\si{s} - 0}
		= \frac{3.00\si{\frac{m}{s}} - 2.00\si{\frac{m}{s}}}{1.00\si{s} - 0}
		= \boxed{1.00\si{\frac{m}{s^2}}}.
	\end{gather*}
	\item \textit{What is the particle's position at the end of this time
	interval (at $t_1 = 1.00\si{s}$)?}
	\begin{align*}
		x(1.00\si{s})
		&= \frac{v_1}{4}(1.00\si{s})^4 + v_0 (1.00\si{s}) + x_0 \\
		&= {
			\frac{1.00\si{\frac{m}{s^4}}}{4}(1.00\si{s})^4
			+ 2.00\si{\frac{m}{s}} (1.00\si{s})
			+ 4.00\si{m}
		} \\
		&= \boxed{6.25\si{m}}
		.
	\end{align*}
\end{enumerate}


\end{document}