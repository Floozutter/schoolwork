\documentclass[12pt, letterpaper]{article}
\usepackage{ifpdf}
\usepackage{mla}
\usepackage{url}

\urlstyle{same}


\begin{document}
\begin{mla}
	{D.}{Choi}
	{Dr. Schroeder}
	{Writing 150, Section 64675}
	{13 November 2020}
	{Exploring the ``Trap'' Debate: Words and Power}


The word ``trap'' has become a rather unique topic of controvery in the online
world. The colloquialism's recent inclusion into the internet-slang vernacular
can be attributed to the online community of English-speaking Japanese pop
culture fans, where it was popularized during the 2000s as a term to describe
androgynous or feminine-presenting fictional characters with male birth sex.
Since then, the word ``trap'' has also been used to label people outside of
fiction as well. However, despite the term's great memetic success, the word
``trap'' is currently at the heart of an ongoing, heated public debate.
Opponents to the word criticize it as a slur for various reasons, such as how
the term carries the negative connotation of deception that can be used to
undermine the identities of trans women, non-binary folks, and cis men alike.
Similarly, proponents of the word defend its use with arguments of their own,
bringing attention to positive aspects of the word such as its good-faith
usage.

The debate on whether ``trap'' is a slur can be difficult to navigate. Often
in discussions, there is no clear consensus on a proper definition for the
word ``trap'', nor is there consensus on what a slur even exactly is. Most
people intuitively understand the act of saying a slur as an expression of
bigotry, but constructing an absolute definition for ``slur'' is nontrivial
due to edge-case pejoratives such as ``pig'', ``sissy'', and ``TERF'' that
blur the line. Because of that, attempting to answer the question, ``Is trap a
slur?'' with the hope of achieving a definitive and unanimously accepted
conclusion is impractical. So instead, this essay explores some of the many
different arguments for why ``trap'' is or is not a slur and their contingent
definitions in order to clarify exactly what is problematic about the word
``trap''. In doing so, we find that normalizing the use of the word ``trap''
in its current form may be especially dangerous for trans people given the
power of the word to promote harmful and dehumanizing social narratives,

First, let us start with the argument that ``trap'' is a slur because of the
word's intrinsic negative connotation of deception. The Oxford English
Dictionary provides a definition for trap formally as, ``A trick by which
someone is misled into acting contrary to their interests or intentions.''
Because ``trap'', the colloquialism, is a polyseme of trap, the formal word,
we can conclude that a connotation of deception does exist. Does this matter?
Well, Davis and McCready in ``The Instability of Slurs'' define a slur using
three criteria: First, a slur by itself ``semantically invokes a complex which
can be used to derogate a particular group''. The authors further define a
``complex'' as external ``historical facts, prejudices, and social
attitudes''. Second, when slurs are used, ``the derogation of that group
functions to subordinate them within some structure of power relations''.
And third, a slur's target group is ``defined by an instrinsic property''.
Let us see if Davis and McCready would classify ``trap'' as a slur under their
definition. The first criterion is fulfilled: Because ``trap'' instrinsically
carries a connotation of deception with regard to gender identity, the word
can be used to derogate all relevant target groups (trans women, non-binary
folks, or cis men) through the complex of gender. The second criterion is also
fulfilled; using ``trap'' derogatorily reinforces the power imbalances
structured by the gender binary. Finally, the third criterion is fulfilled as
well. The groups that ``trap'' can be used to label is defined primarily by
appearance and gender, which are intrinsic properties. Since all three
criteria are fulfilled under Davis and McCready's definition, one could
conclude that ``trap'' is indeed a slur.

However, some counterarguments may dispute this conclusion under Davis and
McCready. One possible response to a definition of this sort is the additional
qualification that a word must always be pejorative, or express contempt, for
it to be a slur. Many users of the word ``trap'' that belong to the Japanese
pop culture fan community may find this requirement intuitive, since some
parts of the community celebrate the term. For example, one fan member
recounts, ``Despite being trans, I'm guilty of having used this term in the
past, being deep in the anime community rather than the social justice
community, I didn't have a lot of exposure to people considering it
problematic. It seemed like a celebratory term of gender non-conformity in
many cases, and so I considered myself of the trope or archetype prevalent in
anime. I would specifically go out of my way to request and watch t-word
anime. I've always hated gender roles and been a fan of gender non-comformity.
So being a part of the `trap' subfandom in anime seemed right up my alley''.
This history of celebratory use muddies the water on the subject greatly.
Justina Diaz-Legaspe's definition of slur, as detailed in ``What is a slur?'',
states that slurs are pejoratives and thus ``are offensive not just when used
in a particular context, but in most or all uses''. If the word ``trap'' can
be used in a neutral or positive manner, then definitions of slur which have
the requirement of pejorativity would not classify ``trap'' as a slur.

The confusion around the word ``trap'' may potentially have to do with the
fact that there is no neutral correlate for the word. Paradigmatic slurs often
have correlate words that lack their derogatory meaning, such as how the
N-word has the correlate ``black''. Lauren Ashwell in ``Gendered Slurs''
states that many gendered slurs like ``slut'', ``bitch'', and ``sissy'' lack
neutral correlates, primarily because their derogatory meaning depends on the
expectation of conformity to some social norm. Just as derogating someone as a
``slut'' may be offensive if a moral value is placed in sexual moderation,
calling someone a ``trap'' may be offensive if a moral value is placed in
adherence to the gender binary. Furthermore, Ashwell argues that a change in
social norms may allow for the reclamation of these words. This framework is
consistent with the idea that subcultures may acceptably use the word ``trap''
if they are doing so in celebration of gender non-conformity, as suggested by
the quoted fan member.

Of course, the fact that some dialogues allow a neutral or positive usage of
the word ``trap'' does not mean that it can't be used in an exceedingly
offensive, slur-like manner. Far-right radical David Ernest Duke said in a
now deleted Tweet that ``Traps are gay,'' in response to Playboy magazine's
first transgender model (The Pedantic Romantic). In doing so, Duke performed
all of the injustices related to saying a slur as described by Davis and
McCready: he derogated not only trans women by invoking the transphobic
complex that trans women aren't women, but also gay people by implying that
heterosexuality is preferable. Furthermore, the term ``trap'' has been
co-opted by pornography creators and consumers alongside existing vulgarities
like ``tranny'' and ``shemale'' as genre labels, which contributes to the
sexualization of transgender people and femininity in general as a social
issue. As of now, LGBTQ+ people still face significant marginalization in
society at large, so the idea that ``trap'' can be used in a non-derogatory
manner in the general case is impracical.

Ultimately, the question of whether ``trap'' is a slur or not is less
important than what may concretely result from encouraging the use of the word
``trap'' socially. Although McCready and Davis differ from Diaz-Legaspe in
their definitions of slur, both of their work is primarily concerned with what
kind of social effects harmful words can have. A key point in McCready and
Davis's definition is the criterion that a slur uses a complex in order to
subordinate a target group. Diaz-Legaspe also dedicates a substantial portion
of her definition to dominance relations between social groups. Indeed, power
dynamics are the essential issue with harmful words, be they slurs or not.
Teresa Marques in ``Beasts in Human Form: How Dangerous Speech Harms'', warns
that ``Speech can harm directly through insult, derogation, and denigration.
And it can harm indirectly by undermining social and moral norms in
surreptitious ways''. Marques names two particular forms of such indirect harm
as guilt attribution and threat construction, which are especially relevant
in the politics surrounding trans people. For example, the language in
anti-transgender bathroom rights propaganda often use a form of threat
construction. In 2016, Former North Carolina Governor Pat McCrory claimed in
response to a nondiscrimination law that protect LGBTQ+ people's right to use
public accommodations that, ``This shift in policy could also create major
public safety issues by putting citizens in possible danger from deviant
actions by individuals taking improper advantage of a bad policy. Also, this
action of allowing a person with male anatomy, for example, to use a female
restroom or locker room will most likely cause immediate State legislative
intervention which I would support as governor'' (Lopez). The use of the word
``trap'' toward trans people may help support the narratives of threat
construction that are being used to dehumanize them. Furthermore, in legal
cases where the defendant has attacked or killed a trans person, a strategy
known as the ``trans panic defense'' has been used to justify the defendant's
violence. When doing so, the defense claims that the trans victim's sexual
advances have provoked the defendant into having a violent reaction, denying
full culpability. This defense is a form of guilt attribution, essentially
arguing that the trans victim should take some blame for the crime of the
perpetrator. Again, the connotation of the word ``trap'' is frighteningly
connected to already existing harmful narratives against trans people.
Normalizing the use of the word ``trap'' toward trans people would have to be
done in the face of all this prior history of dangerous speech.

By considering many of the different ways that the question ``Is trap a
slur?'' can be answered, a more nuanced understanding of the core issues at
hand can be discovered. Overall, this exploration uncovered two key points.
First, about the word ``trap'' itself: A lot about it is problematic.
Analyzing ``trap'' under Davis and McCready's definition reveals its capacity
to function like a slur in full, and thus be used as a tool to subordinate
its target groups in structures of power. Moreover, the word is especially
dangerous for trans people due to existing social narratives that have
employed harmful speech of a similar vein. But despite these undeniable
problems, some members of the target group recall their encounters with the
word ``trap'' as positive and celebratory. Is this in misplaced good-faith,
or is there hope for reclamation, or possibly even an altogether removal of
the derogatory register? It is difficult to say yes now in a global social
climate that is still generally hostile toward LGBTQ+ people such as ours. My
personal, practical recommendation now is to reserve the use of the word
``trap'' to only those who identify with the term. Second, language matters.
Harmful speech can be used as tools to divide and dehumanize people, and
similarly, positive speech can lift up marginalized people in solidarity like
with reclaimed slurs. Don't underestimate the power of words.


\begin{workscited}
	\bibent Ashwell, Lauren. ``Gendered Slurs''.
	\textit{Social theory and practice}, Vol.42 (2), p.228-239, 2016.
	\url{https://uosc.primo.exlibrisgroup.com/permalink/01USC_INST/273cgt/cdi_gale_infotracacademiconefile_A453357045}.

	\bibent Davis, Christopher; McCready, Elin. ``The Instability of Slurs''.
	\textit{Grazer philosophische Studien}, Vol.97 (1), p.63-85, March 2020.
	\url{https://uosc.primo.exlibrisgroup.com/permalink/01USC_INST/273cgt/cdi_crossref_primary_10_1163_18756735_09701005}.

	\bibent Diaz-Legaspe, Justina. ``What is a slur?''.
	\textit{Philosophical Studies}, Vol.177 (5), p.1399-1422, Feb. 18 2019.
	\url{https://uosc.primo.exlibrisgroup.com/permalink/01USC_INST/273cgt/cdi_crossref_primary_10_1007_s11098_019_01259_3}.

	\bibent Lopez, German. ``Anti-transgender bathroom hysteria, explained''.
	\textit{Vox}, Feb. 22, 2017,
	\url{https://www.vox.com/2016/5/5/11592908/transgender-bathroom-laws-rights}.

	\bibent NarrowHipsAreSexy, ``Trap has grown as an incredibly popular
	memetic term in the anime community, to describe gender non-conforming
	or passing characters who are not transgender. I am told this is an
	offensive slur, what should be done to combat this? What replacement term
	could or should be used?''.
	\textit{Reddit}, r/socialjustice101, Mar. 28, 2017,
	\url{https://www.reddit.com/r/socialjustice101/comments/61yy7l/trap_has_grown_as_an_incredibly_popular_memetic/}.

	\bibent Marques, Teresa, ```Beasts in Human Form': How Dangerous Speech Harms''.
	\textit{Araucaria (Triana)}, Vol.21 (42), p.553-584, 2019,
	\url{https://uosc.primo.exlibrisgroup.com/permalink/01USC_INST/273cgt/cdi_dialnet_primary_oai_dialnet_unirioja_es_ART0001374434}.

	\bibent The Pedantic Romantic, ``Traps' Don't Exist And Here's Why''.
	\textit{YouTube}, Nov 26, 2018,
	\url{https://youtu.be/nxeB2AXIG3E}.

	\bibent ``Trap''.
	\textit{Oxford University Press}, 2020,
	\url{https://www.lexico.com/en/definition/trap}.
\end{workscited}


\end{mla}
\end{document}
