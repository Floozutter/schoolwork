\documentclass[12pt, letterpaper]{article}
\usepackage{ifpdf}
\usepackage{mla}

\begin{document}
\begin{mla}
	{D.}{Choi}
	{Dr. Schroeder}
	{Writing 150, Section 64675}
	{13 September 2020}
	{Beautiful Enough to be Valid}

Some microaggressions disguise themselves as compliments. Johanna Ediger,
author of \textit{Cultural Dysphoria} is all too familiar: ``Have you ever
heard a compliment that perfectly outlines your self hatred?'' she asks.
``When I hear `you're pretty for a trans girl', I am unsettled by how good it
feels to hear it.'' This microaggression and its equivalent counterpart,
``you're handsome for a trans man'', serve a hidden function beyond just their
accidental insinuation that trans people are generally less attractive than
cis people. In their intent to validate an individual trans person, these
microaggressions reinforce the dominance of a cisnormative standard for beauty
that is ultimately damaging toward the mental health of the transgender
community as a whole.

First, let's address the opening claim that these microaggressions reinforce a
cisnormative standard for beauty. The plain statement of ``you are pretty''
or ``you are handsome'' without any indication for the subjectivity of
physical attractiveness frames the statement more as an assertion of fact
rather than as an expression of personal taste. This strongly implies that
the speaker's notion of beauty is one that is commonly shared amongst the
majority of people. Furthermore, the qualifier of ``for a trans woman'' or
``for a trans man'' specifies that the notion of beauty does not presuppose
the existence of trans people. Thus, the use of such an assertion as a
compliment implies that a trans person's worth depends on them being
attractive under a cisnormative framework for beauty.

This implied value in conforming to a common, cisnormative standard for beauty
can fuel transphobia, because the validity of trans people are often
questioned on the basis of their appearance. Someone who has never explored
the concept of gender identity sincerely may erroneously conclude that, if a
trans person is not sufficiently masculine or feminine by some traditional
(arbitrary) metric, then they must not \textit{really} be the gender they
identify as. Trans people can also be publically shamed on the basis of how
well their appearance conforms to cisnormative expectations, such as when
\textit{The Independent} published a piece by trans-exclusionary radical
feminist Germaine Greer, ``On Why Sex Change is a Lie''. In her article, Greer
vehemently ridicules a fan of her work who happened to be a trans woman for
the crime of not looking sufficiently feminine. Because of these high stakes
for failing to conform, trans people are essentially held to a higher standard
for cisnormative beauty than cis people.

The high expectations placed upon trans people to conform to cisnormative
beauty standards can be internalized in harmful ways. The concept of
``passing'' (being perceived as cisgender) in the trans community is an
example of this. For trans people, passing can be valued for providing the
material benefit of allowing one to avoid transphobic hostility or
discrimination, but it can also become misinterpreted as an indicator for a
person's self-worth and validity. This can result in misconceptions such as
``I'm not a real man / woman unless I pass''. As mentioned before, the
microaggressions of ``you're pretty for a trans girl'' or ``you're handsome
for a trans man'' are especially likely to reinforce the misattribution of
passing to self-worth because they are intended as compliments, devices with
the social function of improving another's self-esteem.

Now to complete addressing the latter claim, the internalization of
cisnormative beauty standards and the association of beauty with self-worth
can be detrimental to a person's mental health. The study \textit{Searching
Out the Ideal: Awareness of Ideal Body Standards Predicts Lower Global
Self-esteem in Women} by authors Balcetis, Cole, Chelberg, and Alicke found
that the awareness of ideal body standards predicts lower global self-esteem
in women. Generalizing this finding to everyone implies that this trouble is
only compounded for trans people, who are already vulnerable to mental health
issues due to internal factors such as gender dysphoria and external factors
such as discrimination and abuse.

So, if remarking that ``you're pretty for a trans girl'' or ``you're handsome
for a trans man'' to compliment someone is problematic because it reinforces a
cisnormative standard for beauty that is especially harmful for transgender
mental health, what can be said instead? To completely remove beauty as a
topic of discussion from every conversation is unhelpful, because beauty can
and should be something that is positive and empowering. Instead, we should
strive to make our collective definition of beauty more inclusive to
identities outside of the cisgender binary so that trans, non-binary, and
genderqueer people can be considered without shoehorning them into
cisnormative expectations.

But more importantly, a conscious cultural effort should be made to
disassociate an individual's physical attractiveness from their worth as a
person. The media trope of a character being as beautiful as they are virtuous
may possibly have some utility as a storytelling mechanism, but the concept
has no place in the real world. Tear down the idea that a person's beauty has
anything to do with their humanity, and no one will ever need to be beautiful
enough to be valid.

\end{mla}
\end{document}
