\documentclass[12pt, letterpaper]{article}
\usepackage{ifpdf}
\usepackage{mla}

\begin{document}
\begin{mla}
	{D.}{Choi}
	{Dr. Schroeder}
	{Writing 150, Section 64675}
	{12 September 2020}
	{Beautiful Enough to be Valid}

Some microaggressions disguise themselves as compliments. Johanna Ediger,
author of \textit{Cultural Dysphoria} is all too familiar: ``Have you ever
heard a compliment that perfectly outlines your self hatred?'' she asks.
``When I hear `you're pretty for a trans girl', I am unsettled by how good it
feels to hear it.'' This microaggression and its equivalent counterpart,
``you're handsome for a trans man'', serve a hidden function beyond their
accidental insinuation that trans people are generally less attractive than
cis people. In their intent to validate an individual trans person, these
microaggressions reinforce the dominance of a cisnormative standard for beauty
that is ultimately damaging toward the mental health of the transgender
community as a whole.

First, let's address the opening claim that these microaggressions reinforce a
cisnormative standard for beauty. The plain statement of ``you are pretty''
or ``you are handsome'' without any indication for the subjectivity of
physical attractiveness frames the statement more as an assertion of fact
rather than as an expression of personal taste. This strongly implies that
the speaker's notion of beauty is one that is commonly shared amongst the
majority of people. Furthermore, the use of such an assertion as a compliment
implies that being normatively attractive is a valuable trait for one's
self-worth.

This implied value in measuring up to a common standard for beauty can fuel
transphobia, because the validity of trans people are often questioned on the
basis of their appearance. Someone who has never explored the concept of
gender identity in an academic setting may erroneously conclude that, if a
trans person is not sufficiently masculine or feminine by some traditional
(arbitrary) metric, then they must not \textit{really} be the gender they
identify as. Trans people are also often publically shamed on the basis of
their appearance, such as when \textit{The Independent} published a piece by
trans-exclusionary radical feminist Germain Greer, ``On Why Sex Change is a
Lie''. In her article, Greer vehemently ridicules a fan of her work who
happened to be a trans woman for the crime of not looking sufficiently
feminine.

\end{mla}
\end{document}
