\documentclass[12pt, letterpaper]{article}
\usepackage{ifpdf}
\usepackage{mla}

\begin{document}
\begin{mla}
	{D.}{Choi}
	{Dr. Schroeder}
	{WRIT 150 Section 64675}
	{24 August 2020}
	{It Takes Two}

% PROMPT:
% Take a moment to reflect on your own experiences. Then, respond to
% the following questions in an essay of 500 to 1000 words (approximately
% 1 to 2 pages double-spaced):
% To what extent is the identity that you have constructed for yourself
% different from the identity that others might assume for you or impose on
% you? Why does this matter, either in terms of your own life or in terms of
% our society more broadly?

You're not supposed to edit your own Wikipedia page. Although this tiny bit
of internet trivia may seem meaningless, it actually reveals an unfortunate
truth about identity: Although your identity is something that is
unequivocally yours, you don't have the final say in how other people will
choose to define you.

I personally learned about this trouble some time after I fully settled down
since immigrating from South Korea to the United States. I immigrated when I
was fairly young, so I was primarily socialized in American culture. Because
of this, I had little incentive to explore or personally identify with my
South Korean heritage. This would lead to me feeling rather confused whenever
my friends and peers would attribute certain parts of my personality (such as
my bookishness) or my interests (video games) to the fact that I am Korean.
Although I felt that my ethnicity was not a big part of who I am, the people
around me felt otherwise.

The fact that we can, in essence, force an identity upon someone is something
we should be aware of. While the case of me being identified as Korean was
neither particularly inaccurate nor harmful, it did cause me some discomfort.
(There's a sort of dissonance one experiences when they are labelled with
something that they don't identify with.) In other cases, the damage done can
run much deeper. Consider the topic of gender identity: It can incredibly easy
to force your own assumptions of someone's gender and all the cultural baggage
that comes with such an assumption down the throat of someone who is
transgender or questioning.

If we understand that the emotional harm that can result from imposing
identities upon others is not trivial, the natural next question is: How can
we minimize it? Normalizing social practices that allow people to directly
state parts of their self-identities, such as including preferred pronouns
within introductions, is a great first step. However, this is in isolation
doesn't address the problem that many people and aspects of society actively
discourage the public usage of some identities, either by shame, dismissal, or
material disincentives. For example, a transgender person who has stated their
preferred pronouns may have their gender identity consciously rejected by 
transphobic peers. The endgame solution to this problem is to foster a culture
of mutual acceptance that freely encourages expressions of self-identity,
starting with individual relationships. It takes two people to create and
validate an identity, so be a kind partner.

\end{mla}
\end{document}