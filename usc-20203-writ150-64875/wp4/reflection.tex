\documentclass[12pt, letterpaper]{article}
\usepackage{ifpdf}
\usepackage{mla}
\usepackage{url}

\urlstyle{same}


\begin{document}
\begin{mla}
	{D.}{Choi}
	{Dr. Schroeder}
	{Writing 150, Section 64675}
	{17 November 2020}
	{Stress, but Less}


Writing is stressful. There's just something so beautifully nerve-wracking
about having to pour out a part of your soul into text on a page. You think
about the words you're putting down, you think about how those words reflect
on you as a writer, you think about your goals, your life, your self, and then
you stop writing because your brain's overheated from thinking too much and
you have to take a break. It can be utterly draining to work through a
productive writing session, which is sort of a problem when you have a
semester's worth of essays to get done.

I've had my fair share of burnout during my time in Writing 150, and because
of that, I learned firsthand just how demoralizing it can feel to lose your
stride as a writer. But on the upside, difficult experiences are often also
great teachers, so I picked up a couple of lessons on how to manage stress
while writing for my trouble. Thematically, what I learned share one
principle: diffuse your cognitive load.

Ancillary assignments exist for a reason. Whenever I felt overloaded during
the semester, I would skip some of the smaller assignments that accompany the
main writing projects with the reasoning that I would sacrifice a bit of my
overall grade for some extra time to regain footing. Terrible idea. Instead of
feeling better, I piled on more stress by backloading the cognitive work I had
do for the final essays closer to their deadlines. Nothing screams ``panic!''
louder than a paper due in the morning when you can't figure out what you're
supposed to write next. The whole point of the ancillary assignments was to
build up a secure conceptual framework for each essay, which is an incredible
relief to rely on during the writing process. I remember that my experience
with the second writing project was so much healthier than with the first,
specifically because I completed more ancillary assignments before I started
work on the essay. And not only did it feel more stable while writing, I think
my second essay ended up vastly better in structure because of it.

In a similar vein, it's also crucial to relax and immerse yourself with the
material you're writing about early. Just as I avoided doing ancillary work in
a futile attempt to lower stress, I also tried to put off engaging with any
other sources for as long as possible. This really didn't make much sense at
all, mainly because everything I wrote about in this class were topics that I
were (and still am) genuinely interested in. My issue was a matter of
perspective: When I was leafing through Wikipedia articles and forum posts
about machine learning online, I was having fun and killing time. But when I
was doing the exact same thing as research for my Writing Project 2 essay on
how facial recognition is vulnerable to systemic racism, it suddenly became
work. Stressful, painful work. A large part of why this emotional shift
occurs is probably because of the looming expectation that I would have to
integrate whatever I'm currently reading into an essay that I haven't even
figured out yet. This is a pretty unfortunate phenomenon that I experience
not only when writing, but also when doing other hobbies after introducing
some sort of anxiety-inducing end goal. For example, when I'm programming for
myself, it's often a peaceful and relaxing affair, but when I'm programming
for a school assignment, the added pressure can sour the entire experience.
What helps for me is to be aware of when this happens, remind myself that I
intrinsically enjoy what I'm supposed to be doing, and try to forget about
any extraneous expectations. Not only will avoiding a topic you love that you
have an essay on make the writing process more rushed and stressful, it will
also make you sad.

Finally, as a simple counterpart to the previous idea, choose a topic that
you're interested in. Writing about something you care about can help reduce
the cognitive load of research, since you're likely to already be acquainted
with resources, and also the load of analysis, since you may have already
formed an opinion by intuition. My Writing Project 3 suffered a bit because I
simply started with less domain knowledge on the topic, class struggle, than I
had for my other essays. Furthermore, engaging with material you care about
can make the experience of writing more rewarding. For instance, I had a much
more positive experience working on Writing Project 4 because I found the
topics I chose, linguistics and queer studies, very fulfilling to write about.

The three lessons I learned of doing supporting work, engaging with the
material early, and choosing a topic of interest are all related by how they
help diffuse the effort required in writing the final draft. By working on
ancillary assignments related to argument and structure, you can create a
framework for your essay that you can depend on while writing. Immersing
yourself in other sources helps in a similar way by getting you intimately
familiar with the topic. And choosing a topic you care about helps in dual
by leveraging your unique prior experience, and by making the writing process
more rewarding to distract you from the stress. Although I've been far from a
perfect student or writer as a member of this class, I had an incredibly
gratifying experience learning these lessons and more over the course of this
semester. Writing is stressful, but it's also rewarding like nothing else, so
I hope I can take what I learned to heart.


\end{mla}
\end{document}
