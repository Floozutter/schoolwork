\documentclass[12pt, letterpaper]{article}
\usepackage{ifpdf}
\usepackage{mla}
\usepackage{url}

\urlstyle{same}


\begin{document}
\begin{mla}
	{D.}{Choi}
	{Dr. Schroeder}
	{Writing 150, Section 64675}
	{13 November 2020}
	{The ``Trap'' Debate: Words and Power}


[Introduction (Context)]
In many online spaces now, the word ``trap'' has become a rather unique topic
of controversy. The word's recent inclusion into the internet-slang vernacular
began during the 2000s in the online community of English-speaking fans of
Japanese pop culture products such as anime and manga, where it was
popularized as a term to describe androgynous or feminine-presenting fictional
characters with male birth sex. Since then, the word ``trap'' was semantically
widened so that it is now often used to describe people outside of fiction as
well. However, despite the term's great memetic success, it is not without its
problems. [Opponents to the word ``trap'' criticize the term for various
reasons, such as its use to indiscriminately lump together people of disparate
gender like trans women, nonbinary folk, and cis men under a catch-all label
that erases the nuances in their identities.]/[Opponents to the word ``trap''
criticize it as a slur for various reasons, such as how the term carries the
negative connotation of deception that can be used to undermine the identities
of trans women, non-binary folk, and cis men alike.] Similarly, proponents of
the word defend its use with arguments of their own, bringing attention to the
term's often good-faith usage and semantic utility.

[Introduction (Thesis Statement)]
To navigate the debate about the word ``trap'' in an informed manner, a number
of questions must first be addressed. Is ``trap'' a slur? What good can result
from encouraging the word's use? What bad? By examining the debate along these
perspectives, this essay hopes to prove that the negatives of the word
``trap'' are significant enough to warrant caution against its current
colloquial use, specifically because of the power imbalance involved between
the word's frequent users and the social groups that are its frequent targets.


\begin{workscited}
\end{workscited}


\end{mla}
\end{document}
