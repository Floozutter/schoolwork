\documentclass[12pt, letterpaper]{article}
\usepackage{ifpdf}
\usepackage{mla}

\newcommand{\comment}[1]{}

\begin{document}
\begin{mla}
	{D.}{Choi}
	{Dr. Schroeder}
	{Writing 150, Section 64675}
	{23 October 2020}
	{Writing Project 3}


\textbf{Introduction 1 - Context:} \textit{Snowpiercer} begins with a grim
outlook for humanity. In its premise, the impending consequences of global
warming due to climate change have come so irrefutably dire that a
global-scale climate engineering project was used to cool down the Earth's
surface. This inadvertently plunged humanity into an ice age where its
remaining survivors must weather the cold in a perpetually-running train named
the Snowpiercer. Within the Snowpiercer, society was structured into a class
system by section. Those at the front section are able to enjoy the train's
many technological luxuries, while the majority at the tail are coerced into
privation by armed guards.

\textbf{Introduction 2 - Thesis:} At its core, \textit{Snowpiercer} is not shy
about being a critique of capitalism. The film's narrative is essentially
about the Snowpiercer's tail section protagonists fighting against their class
oppression, a system which was born out of which survivors could afford
tickets to the front section. Furthermore, the film's bleak environment was
caused by climate change, a problem that was caused by industrial capitalism
in the first place. What makes \textit{Snowpiercer}'s critique particularly
interesting is in how its central conflict is resolved by the protagonists
breaking free from the train's class hierarchy, rather than restructuring the
hierarchy from within. In doing so, the film uses its overall theme of class
struggle alongside its subversive narrative structure to rebut the notion that
unjust hierarchies can be fixed from within their systems. Through this
approach, \textit{Snowpiercer} advocates for a revolutionary solution to the
problems of capitalism it exemplifies.


\begin{workscited}
\end{workscited}


\end{mla}
\end{document}
