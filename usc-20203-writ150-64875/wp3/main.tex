\documentclass[12pt, letterpaper]{article}
\usepackage{ifpdf}
\usepackage{mla}

\newcommand{\comment}[1]{}

\begin{document}
\begin{mla}
	{D.}{Choi}
	{Dr. Schroeder}
	{Writing 150, Section 64675}
	{27 October 2020}
	{Subverting Structures in \textit{Snowpiercer}}


\textit{Snowpiercer} is a film with a grim outlook for humanity. In its
backstory, the impending consequences of global warming due to climate change
have become so irrefutably dire that a wide-scale climate engineering project
was used to cool down the Earth's surface. This inadvertently plunged humanity
into an ice age where its remaining survivors must weather the cold in a
perpetually-running train named the Snowpiercer. Within the Snowpiercer,
society was structured into a class system by section, where the few at the
front section are able to enjoy the train's many technological luxuries, while
the majority at the tail are coerced into privation by armed guards.

Even from its premise, \textit{Snowpiercer} is clear about intending to make a
statement about capitalism. The film's backdrop of industrial climate change
alongside the narrative's main driving conflict of class struggle is prime
material for exploring and criticizing capitalism. However, the place where
\textit{Snowpiercer}'s messaging truly shines in its ending: By subverting the
tail-to-front structure of its narrative and central conflict in its
resolution, \textit{Snowpiercer} rebuts the notion that capitalism's issues
can be resolved internally. Through this, \textit{Snowpiercer} advocates for
a revolutionary solution to the problems of capitalism.

To see how \textit{Snowpiercer}'s resolution subverts expectations, it is
important to recognize what expectations the film sets up in the first place.
\textit{Snowpiercer} starts its narrative by establishing the disparity in the
quality-of-life between the train's front and tail sections as the motivating
conflict for the tail section protagonists. The story progresses from
tail-to-front as Curtis Everret, the leader of the tail section, and his
crew go forward from car to car. They first do this with the intent to
negotiate better living conditions with the front of the train, but they later
resolve to throw a coup and dispatch with the train's owner, Wilford, who
resides at the head of the train where the engine is. This narrative structure
is actually also reinforced by the film's cinematography. The film form
analysis channel Every Frame a Painting describes the pattern in their video
essay ``\textit{Snowpiercer} - Left or Right'', ``\textit{Snowpiercer} depends
on one simple rule: Camera left is the back of the train, while camera right
is the front. We gradually move from left to right''. This all culminates in
the establishment of a binary possibility for the story's ending: Either
Curtis and the tail section will reach the far-right and succeed in their coup
at the train's engine, or they will be cut short in the middle and fail.

This binary is eventually subverted when Curtis finally manages to reach the
front of the Snowpiercer. He encounters Wilford, and it is revealed that for
the train's ecosystem to continue functioning after his coup, Curtis must
replace Wilford as the train's head. It is also revealed that the train's
``Eternal Engine'' employs the labor of child slaves. Curtis subsequently
rejects Wilford's position, thus denying both of the two possibilities of
either restructuring the hierarchy but maintaining the system or preserving
the existing status quo. Instead, Curtis and his ally Namgoong, who is
motivated primarily by his interest in the frozen landscape outside, detonate
an explosion at the side of the Snowpiercer. This destroys the train, leaving
the surviving passengers to brave the unexplored world external to the
Snowpiercer, free from its hierarchy. Similarly, the camera is finally allowed
to move orthogonally to its established side-scrolling pattern. Although the
entirety of the film's narrative and cinematography was constructed within the
linear tail-to-front system of the Snowpiercer, their tension could only truly
be resolved by dismantling the system and looking outward.

On the basis that the Snowpiercer is representative of capitalism (in its
unjust hierarchy, exploitation of child labor, and its inability to act as a
solution to climate change without presupposing it), the destruction of the
Snowpiercer being a necessary condition to resolve the narrative's tensions
within the film echoes the idea that capitalism is doomed to fail. This
concept is a crucial result of crisis theory in Marxist economics. In ``Crisis
of What: The End of Capitalism or Another Systemic Cycle of Capitalist
Accumulation?'' Christopher Chase-Dunn builds off of crisis theory and poses
the question of whether capitalism will cease under an upcoming crisis or
reset itself once the crisis resolves. The author personally predicts the
latter, ``Both a new stage of capitalism and a qualitative systemic
transformation to some form of socialism are possible within the next several
decades, but a new systemic cycle of capitalism is probably more likely,'' a
viewpoint that \textit{Snowpiercer} shares. The expected ending presented by
the narrative's binary was based on the idea that Curtis would helm the
Snowpiercer once his coup against Wilford succeeded, continuing the cycle of
class division and oppression anew. Thus, the film implies that the rejection
of a new cycle as allowed by Curtis's morality and the destruction of the
Snowpiercer as allowed by Namgoong's outward perspective was an exceptional
result. Because \textit{Snowpiercer} assumes that the default behavior of
capitalism under crisis is to reset itself, the film imparts a message that
is a both a warning and a call to action: An unjust system like capitalism
will maintain itself until directly overthrown.

So if \textit{Snowpiercer} advocates for dismantling capitalism, what kind of
praxis does the film recommend? From the film's depiction of the tail section
organizing to overthrow the train's class hierarchy without a state, it can be
argued that the film suggests a form of autonomism, where direct and
peer-to-peer action is emphasized. Furthermore, ``Autonomism in Theory and
Practice'' argues that autonomists can be somewhat distinguished from
classical Marxists by their dismissal toward reform: ``Some autonomists reject
any and all struggles around reform, of course, perhaps on the grounds that
global economic conditions should be allowed to worsen in order to instigate
more global insurrection. Such arguments truly take autonomists beyond the
range of Marxism, we would argue.'' Curtis may have had this sort of
accelerationist mindset when he refused to further maintain the Snowpiercer
like Wilford had, and instead chose to accept Namgoong's idealism and
destroyed the train. However, ``Not All are Aboard: Decolonizing Exodus in
Joon-ho Bong’s \textit{Snowpiercer}'' argues specifically that the film can
not adequately be described using only an autonomist framework. The article
argues that decolonialism offers a more complete perspective, primarily
because the autonomist interpretation ``overlooks how the train sections
represent not only class formations within capitalism, but also racial
formations under colonialism''. This is a valid argument, but for the sake of
restricting this analysis to the scope of class, autonomism is a sufficient
theory to map \textit{Snowpiercer}'s messages onto.

The strategy \textit{Snowpiercer} used to convey its message about capitalism
was as subversive as its political ideology. The film fully utilized its
medium in both storytelling and cinematography to hammer in the train as a
dominant, established structure so that it seemed as if all possible narrative
paths revolved around the Snowpiercer. The film then blew through that system
to show its audience that a solution existed beyond it. Even if the audience
may disagree with the particular form of political action that the film
supports (autonomism), \textit{Snowpiercer} does the valuable work of
introducing new possibilities to a modern political climate that is primarily
dominated by neoliberalism.


\begin{workscited}
\end{workscited}


\end{mla}
\end{document}
