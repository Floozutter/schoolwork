\documentclass[12pt, letterpaper]{article}
\usepackage{ifpdf}
\usepackage{mla}

\newcommand{\comment}[1]{}

\begin{document}
\begin{mla}
	{D.}{Choi}
	{Dr. Schroeder}
	{Writing 150, Section 64675}
	{27 October 2020}
	{Subverting Structures in \textit{Snowpiercer}}


\textit{Snowpiercer} is a film with a grim outlook for humanity. In its
backstory, the impending consequences of global warming due to climate change
have become so irrefutably dire that a wide-scale climate engineering project
was used to cool down the Earth's surface. This inadvertently plunged humanity
into an ice age where its remaining survivors must weather the cold in a
perpetually-running train named the Snowpiercer. Within the Snowpiercer,
society was structured into a class system by section, where the few at the
front section are able to enjoy the train's many technological luxuries, while
the majority at the tail are coerced into privation by armed guards.

Even from its premise, \textit{Snowpiercer} is clear about intending to make a
statement about capitalism. The film's backdrop of industrial climate change
alongside the narrative's main driving conflict of class struggle is prime
material for exploring and criticizing capitalism. However, the place where
\textit{Snowpiercer}'s messaging truly shines in its ending: By subverting the
tail-to-front structure of its narrative and central conflict in its
resolution, \textit{Snowpiercer} rebuts the notion that unjust hierarchies can
be fixed within their systems. Through this, \textit{Snowpiercer} advocates
for a revolutionary solution to the problems of capitalism.


\begin{workscited}
\end{workscited}


\end{mla}
\end{document}
