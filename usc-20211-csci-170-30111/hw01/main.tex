% preamble
\documentclass[12pt]{article}
\usepackage[margin=1.25in]{geometry}
\usepackage[fleqn]{amsmath}
\usepackage{enumitem}
\usepackage{textcomp}
\usepackage{gensymb}
\usepackage{amsfonts}
%\usepackage{tikz}  % figures
%\usepackage{subfiles}  % subfiles

%% formatting
%\pagenumbering{gobble}  % single-page document

%% title
\newcommand{\homework}{1}
\newcommand{\name}{D. Choi}
\newcommand{\duedate}{2021-01-29}
\title{\vspace{-2\baselineskip}CSCI 170 -- Homework \homework}
\author{\name}
\date{\duedate}

%% commands
\newcommand{\qa}{\\[0.5\baselineskip]}  % space between question and answer
\newcommand{\ps}{\mathcal{P}}

% document
\begin{document}
\maketitle

\paragraph{1.}
\textit{Suppose $A$ and $B$ are finite sets.}
\begin{enumerate}[label=\textbf{\alph*.}]
    \item \textit{Compare the two quantities $|\ps(A \times B)|$ and $|\ps(A)| |\ps(B)|$.
    Under what circumstances is the ratio $\frac{|\ps(A \times B)|}{|\ps(A)| |\ps(B)|}$
    equal to $1$, less than $1$, or greater than $1$?} \qa
    Given that $|X \times Y| = |X| \cdot |Y|$ and $|\ps(X)| = 2^{|X|}$,
    \begin{gather*}
        |\ps(A \times B)| = 2^{|A \times B|} = 2^{|A| |B|}, \\
        |\ps(A)| |\ps(B)| = 2^{|A|} 2^{|B|} = 2^{|A| + |B|}.
    \end{gather*}
    Thus, the ratio can be simplified to
    \begin{equation*}
        \frac{|\ps(A \times B)|}{|\ps(A)| |\ps(B)|} = \frac{2^{|A| |B|}}{2^{|A| + |B|}}.
    \end{equation*}
    The numerator and denominator of this simplified ratio are both exponentiations that share the
    same base $2$. Consequently, the ratio will be
    \begin{itemize}
        \item equal to $1$ when $|A| |B| = |A| + |B|$,
        \item less than $1$ when $|A| |B| < |A| + |B|$, and
        \item greater than $1$ when $|A| |B| > |A| + |B|$.
    \end{itemize}
    \item \textit{Does $(A - B) \cap (B - A) = \{\}$?} \qa
    \boxed{\text{Yes}}, $(A - B) \cap (B - A)
    = (A \cap \overline{B}) \cap (\overline{A} \cap B)
    = A \cap \overline{A} \cap B \cap \overline{B}
    = \{\}.$
\end{enumerate}

\paragraph{2.}
\textit{Are these statements true or false and why?}
\begin{enumerate}[label=\textbf{\alph*.}]
    \item $\{\} = \{\{\}\}$? \boxed{\text{False}}, $|\{\}| = 0$ but $|\{\{\}\}| = 1$.
    \item $\{\} = \{0\}$? \boxed{\text{False}}, $|\{\}| = 0$ but $|\{0\}| = 1$.
    \item $|\{\{\}\}| = 0$? \boxed{\text{False}}, $\{\} \in \{\{\}\}$.
    \item $|\ps(\{\})| = 0$? \boxed{\text{False}}, $\{\} \in \ps(\{\})$.
    \item $\{\} \in \{\}$? \boxed{\text{False}}, the empty set contains no elements.
    \item $\{\} = \{x \in \mathcal{N}: x <= 0 \text{ and } x > 0\}$? \boxed{\text{True}}, the
    conditions for each element of the set $x$ are negations of each other.
\end{enumerate}

\paragraph{3.}
% \textit{?} \qa
The intervals $[1, 20], [21, 40], [41, 60]$ fully cover $[1, 60]$ with no overlap. Consequently,
any choice of $4$ distinct integers from $[1, 60]$ will result in at least one of these intervals
containing more than one chosen integer, due to the pigeonhole principle. The maximum difference
of any two numbers both in any one of these intervals is $19$.

\paragraph{4.}
% \textit{?} \qa
\begin{enumerate}[label=\textbf{\alph*.}]
    \item % \textit{?} \qa
    Listing the elements of $C$ gives
    \begin{align*}
        C = \{ & \\
        & \{3, 4\}, \\
        & \{2\}, \\
        & \{1, 4\}, \\
        & \{2, 4, 5\}, \\
        & \{2, 5\}, \\
        & \{3\}, \\
        & \{2, 4\}, \\
        & \{1\}, \\
        & \{1, 5\}, \\
        & \{1, 4, 5\}, \\
        & \{3, 5\}, \\
        & \{3, 4, 5\} \\
        \}.
    \end{align*}
    Thus, by counting, \boxed{|C| = 12}.
    \item % \textit{?}
    \begin{enumerate}[label=\textbf{\roman*.}]
        \item $\boxed{(1, 4, 9)} \in (A \times B \times C)$.
        \item $\boxed{(4, 1, 9)} \in (B \times A \times C)$.
    \end{enumerate}
\end{enumerate}

\paragraph{5.}
% \textit{?} \qa
\boxed{\text{True}}, if $A \times C = B \times C$ then $a \in B$ for every $a$ in
$A \times C = \{(a, c) | a \in A \text{ and } c \in C\}$, and vice versa. Thus,
$A \subseteq B$ and $B \subseteq A$, showing that $A = B$.

\end{document}
