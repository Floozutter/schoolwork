% Preamble.
\documentclass[12pt]{article}
\usepackage[margin=1.25in]{geometry}
\usepackage[fleqn]{amsmath}
\usepackage{textcomp}
\usepackage{gensymb}
\usepackage{amsfonts}
\usepackage{enumitem}
%\usepackage{tikz}  % Include for figures.
%\usepackage{subfiles}  % Include for subfiles.

%% Title macros.
\newcommand{\EXAMNUM}{1}
\newcommand{\NAME}{D. Choi}
\newcommand{\DATE}{2020-06-15}

\title{\vspace{-2\baselineskip}MATH 225 - Exam \#\EXAMNUM}
\author{\NAME}
\date{\DATE}

%% Formatting options.
%\pagenumbering{gobble}  % Include for single-page document.

% Macros.
%% Norm.
\newcommand{\norm}[1]{\left\lVert#1\right\rVert}
%% Rotation and reflection matrices.
\newcommand{\rota}[1]{\text{Rot}(#1 \degree)}
\newcommand{\refl}[1]{\text{Ref}(#1 \degree)}
%% Augmented matrix.
% https://tex.stackexchange.com/a/2238
\newenvironment{amatrix}[1]{%
  \left(\begin{array}{@{}*{#1}{c}|c@{}}
}{%
  \end{array}\right)
}


% Document.
\begin{document}
\maketitle

\section*{1.}
\textit{Let $A$ be the matrix associated with the orthogonal projection onto
the plane $x + y - 3z = 0$.}
\begin{enumerate}[label=(\alph*)]
	\item \textit{Find $A$.} \\[\baselineskip]
	The equation for the plane $x + y - 3z = 0$ can be rewritten as
	\begin{equation*}
		\begin{pmatrix} 1 \\ 1 \\ -3 \end{pmatrix}
		\cdot
		\begin{pmatrix} x \\ y \\ z \end{pmatrix}
		= \vec{n} \cdot \vec{v}
		= 0
		.
	\end{equation*}
	In this form, it is apparent that $\vec{n}$ is a normal vector to the plane.
	\\
	For any vector $\vec{v}$, the vector rejection of $\vec{v}$ from $\vec{n}$
	(which is $\vec{v} - \text{proj}_{\vec{n}}\vec{v}$) will be the orthographic
	projection of $\vec{v}$ onto the plane. Thus, for the standard basis unit
	vectors,
	\begin{gather*}
		\begin{pmatrix} 1 \\ 0 \\ 0 \end{pmatrix}
		\overset{A}{\longrightarrow}
		\begin{pmatrix} 1 \\ 0 \\ 0 \end{pmatrix}
		-
		\frac{
			\begin{pmatrix} 1 \\ 0 \\ 0 \end{pmatrix}
			\cdot
			\begin{pmatrix} 1 \\ 1 \\ -3 \end{pmatrix}
		}{
			\norm{\begin{pmatrix} 1 \\ 1 \\ -3 \end{pmatrix}}^2
		}
		\begin{pmatrix} 1 \\ 1 \\ -3 \end{pmatrix}
		=
		\begin{pmatrix} 1 \\ 0 \\ 0 \end{pmatrix}
		-
		\frac{1}{11}
		\begin{pmatrix} 1 \\ 1 \\ -3 \end{pmatrix}
		=
		\begin{pmatrix}
			\frac{10}{11} \\ -\frac{1}{11} \\ \frac{3}{11}
		\end{pmatrix}
		\\
		\begin{pmatrix} 0 \\ 1 \\ 0 \end{pmatrix}
		\overset{A}{\longrightarrow}
		\ldots
		=
		\begin{pmatrix} 0 \\ 1 \\ 0 \end{pmatrix}
		-
		\frac{1}{11}
		\begin{pmatrix} 1 \\ 1 \\ -3 \end{pmatrix}
		=
		\begin{pmatrix}
			-\frac{1}{11} \\ \frac{10}{11} \\ \frac{3}{11}
		\end{pmatrix}
		\\
		\begin{pmatrix} 0 \\ 0 \\ 1 \end{pmatrix}
		\overset{A}{\longrightarrow}
		\ldots
		=
		\begin{pmatrix} 0 \\ 0 \\ 1 \end{pmatrix}
		+
		\frac{3}{11}
		\begin{pmatrix} 1 \\ 1 \\ -3 \end{pmatrix}
		=
		\begin{pmatrix}
			\frac{3}{11} \\ \frac{3}{11} \\ \frac{2}{11}
		\end{pmatrix}
	\end{gather*}
	And so,
	\begin{equation*}
		\boxed{
			A =
			\frac{1}{11}
			\begin{pmatrix}
				10 & -1 & 3 \\
				-1 & 10 & 3 \\
				-3 & 3 & 2
			\end{pmatrix}
		}
		.
	\end{equation*}
	
	\item \textit{Let $\{\vec{v}_1, \vec{v}_2, \vec{v}_3\}$ be the columns of
	$A$. Are $\{\vec{v}_1, \vec{v}_2, \vec{v}_3\}$ linearly independent? Why
	or why not?}\\[\baselineskip]
	$\{\vec{v}_1, \vec{v}_2, \vec{v}_3\}$ cannot be linearly independent
	because $A$ is a projection matrix from $\mathbb{R}^3$ onto a plane. The
	range of such a projection matrix does not include all of $\mathbb{R}^3$,
	meaning that the column space of $A$ does not include all of
	$\mathbb{R}^3$. The linear dependence of
	$\{\vec{v}_1, \vec{v}_2, \vec{v}_3\}$ can also be verified by computing
	the determinant of $A$, which should be $0$.
\end{enumerate}

\section*{2.}
\textit{Let $B$ be the matrix associated with the reflection across the plane
$x = 0$.}
\begin{enumerate}[label=(\alph*)]
	\item \textit{Let $\{\vec{w}_1, \vec{w}_2, \vec{w}_3\}$ be the columns of
	$B$. What is span$\{\vec{w}_1, \vec{w}_2, \vec{w}_3\}$? Explain your
	reasoning.} \\[\baselineskip]
	The span of $\{\vec{w}_1, \vec{w}_2, \vec{w}_3\}$ is $\mathbb{R}^3$.
	Such a reflection matrix $B$ is invertible. An invertible $n \times n$
	matrix over $\mathbb{R}$ has column space $\mathbb{R}^n$ because such a
	matrix must be bijective. Thus, the span of
	$\{\vec{w}_1, \vec{w}_2, \vec{w}_3\}$ is $\mathbb{R}^3$.
	
	\item \textit{Find all solutions to the system $AB\vec{x} = (1, 2, 1)$.}
	\\[\baselineskip]
	Solutions to $A\vec{v} = (1, 2, 1)$ are
	\begin{equation*}
		\vec{v} =
		\begin{pmatrix} 1 \\ 2 \\ 1 \end{pmatrix}
		+ s \begin{pmatrix} 1 \\ 1 \\ -3 \end{pmatrix}
	\end{equation*}
	Thus,
	\begin{equation*}
		B\vec{x} = \vec{v} =
		\begin{pmatrix} 1 \\ 2 \\ 1 \end{pmatrix}
		+ s \begin{pmatrix} 1 \\ 1 \\ -3 \end{pmatrix}
		.
	\end{equation*}
	Because $B$ is a reflection matrix, $B = B^{-1}$, and so
	\begin{align*}
		\vec{x} = B(
		\begin{pmatrix} 1 \\ 2 \\ 1 \end{pmatrix}
		+ s \begin{pmatrix} 1 \\ 1 \\ -3 \end{pmatrix}
		)
		=
		\boxed{
			\begin{pmatrix} -1 \\ 2 \\ 1 \end{pmatrix}
			+ B (s \begin{pmatrix} 1 \\ 1 \\ -3 \end{pmatrix})
		}
		.
	\end{align*}
\end{enumerate}

\section*{3.}
\textit{The reduced row echelon form of the left matrix $M$ is the right
matrix.}
\begin{equation*}
	\begin{pmatrix}
		1 & 1 & -1 & 1 \\
		2 & 1 & -1 & 3 \\
		1 & 4 & -4 & -2 \\
		2 & 0 & 1 & 2
	\end{pmatrix}
	\xrightarrow{\text{RREF}}
	\begin{pmatrix}
		1 & 0 & 0 & 2 \\
		0 & 1 & 0 & -3 \\
		0 & 0 & 1 & -2 \\
		0 & 0 & 0 & 0
	\end{pmatrix}
\end{equation*}
\begin{enumerate}[label=(\alph*)]
	\item \textit{Find the solution(s) to $M\vec{x} = 0$.}
	\begin{equation*}
		\begin{amatrix}{4}
			1 & 0 & 0 & 2 & 0 \\
			0 & 1 & 0 & -3 & 0 \\
			0 & 0 & 1 & -2 & 0 \\
			0 & 0 & 0 & 0 & 0
		\end{amatrix}
	\end{equation*}
	\begin{gather*}
		x + 2w = 0 \\
		y - 3w = 0 \\
		z - 2w = 0
	\end{gather*}
	Solutions to the system are of the form
	\begin{equation*}
		\begin{pmatrix}
			-2w \\ 3w \\ 2w \\ w
		\end{pmatrix}
		=
		\boxed{
			\begin{pmatrix}
				-2 \\ 3 \\ 2 \\ 1
			\end{pmatrix}
			w
		}
		.
	\end{equation*}
	
	\item \textit{Would it be easy to solve $M\vec{x} = (1, 1, 1, 0)$ without
	any further information?} \\[\baselineskip]
	Yes.
	\begin{gather*}
		x + 2w = 1 \\
		y - 3w = 1 \\
		z - 2w = 1
	\end{gather*}
	Solutions to the system are of the form
	\begin{equation*}
		\begin{pmatrix}
			1 -2w \\ 1 + 3w \\ 1 + 2w \\ w
		\end{pmatrix}
		=
		\boxed{
			\begin{pmatrix}
				1 \\ 1 \\ 1 \\ 0
			\end{pmatrix}
			+
			\begin{pmatrix}
				-2 \\ 3 \\ 2 \\ 1
			\end{pmatrix}
			w
		}
		.
	\end{equation*}
	
	\item \textit{Write the first column of $M$ as a linear combination of
	the last $3$ columns.}
	\begin{equation*}
		\begin{pmatrix}
			1 & -1 & 1 \\
			1 & -1 & 3 \\
			4 & -4 & -2 \\
			0 & 1 & 2
		\end{pmatrix}
		\begin{pmatrix} x \\ y \\ z \end{pmatrix}
		=
		\begin{pmatrix}
			1 \\ 2 \\ 1 \\ 2
		\end{pmatrix}
	\end{equation*}
	Plugging this into GNU Octave, a solution is
	\begin{equation*}
		\begin{pmatrix} x \\ y \\ z \end{pmatrix}
		=
		\begin{pmatrix} 1.5 \\ 1 \\ 0.5 \end{pmatrix}
		.
	\end{equation*}
	To verify,
	\begin{equation*}
		\boxed{
			1.5 \begin{pmatrix} 1 \\ 1 \\ 4 \\ 0 \end{pmatrix}
			+
			\begin{pmatrix} -1 \\ -1 \\ -4 \\ 1 \end{pmatrix}
			+
			0.5 \begin{pmatrix} 1 \\ 3 \\ -2 \\ 2 \end{pmatrix}
			=
			\begin{pmatrix} 1 \\ 2 \\ 1 \\ 2 \end{pmatrix}
		}
		.
	\end{equation*}
\end{enumerate}

\section*{4.}
\textit{Let $N$ be a $100 \times 100$ matrix with the following pattern: \\
\indent Everything on and above the diagonal is 1. \\
\indent Everything below the diagonal is 0.}
\begin{enumerate}[label=(\alph*)]
	\item \textit{What is the first column of $N^{-1}$? \\
	Hint: Can you solve $N\vec{x} = (1, 0, 0, ..., 0)$?}
	\begin{equation*}
		\boxed{\vec{x} = (1, 0, 0, ..., 0)}
	\end{equation*}
	
	\item \textit{What is the third column of $N^{-1}$? \\
	Hint: Can you guess what a solution to
	$N\vec{x} = (0, 0, 1, 0, 0, ..., 0)$ is by playing with linear
	combinations of columns of $N$?}
	\begin{equation*}
		\boxed{\vec{x} = (0, -1, 1, 0, 0, ..., 0)}
	\end{equation*}
	
	\item \textit{What is $N N^{-1}$?} \\[\baselineskip]
	is this a trick question i'm scared
	\begin{equation*}
		\boxed{N N^{-1} = I_{100}}
	\end{equation*}
\end{enumerate}

\section*{5.}
\textit{Let $C$, $D$, and $E$ be $2 \times 2$ matrices that do the
following: \\
\indent C: $\rota{45}$ \\
\indent D: $\rota{-60}$ \\
\indent E: $\refl{30}$}
\begin{enumerate}[label=(\alph*)]
	\item \textit{Find $DE^7C$.}
	\begin{align*}
		DE^7C &= DEC = \rota{-60} (\refl{30} (\rota{45})) \\
		&= \rota{-60} (\refl{7.5}) = \boxed{\refl{-22.5}}
	\end{align*}
	
	\item \textit{Find $(CE^8D)^{-1}$.}
	\begin{equation*}
		(CE^8D)^{1} = (CD)^{1} =
		(\rota{45} \rota{-60})^{-1} =
		(\rota{-15})^{-1} =
		\boxed{\rota{15}}
	\end{equation*}
\end{enumerate}

\section*{6.}
\textit{Let the $2 \times 2$ matrix $U$ have columns $\vec{v}$ and $\vec{w}$:
$U = (\vec{v} | \vec{w})$. \\ {[Diagram omitted.]}}
\begin{equation*}
	\vec{v} = \begin{pmatrix} 3 \\ 1 \end{pmatrix}
	,
	\vec{w} = \begin{pmatrix} 1 \\ 3 \end{pmatrix}
	,
	\vec{b} = \begin{pmatrix} 8 \\ 8 \end{pmatrix}
\end{equation*}
\begin{enumerate}[label=(\alph*)]
	\item \textit{Solve $U\vec{x} = \vec{b}$.}
	\begin{gather*}
		\begin{pmatrix}
			3 & 1 \\
			1 & 3
		\end{pmatrix}
		\begin{pmatrix} x \\ y \end{pmatrix}
		=
		\begin{pmatrix} 8 \\ 8 \end{pmatrix}
		\\
		\boxed{
			\begin{pmatrix} x \\ y \end{pmatrix}
			=
			\begin{pmatrix} 2 \\ 2 \end{pmatrix}
		}
	\end{gather*}
	
	\item \textit{Find $U^{-1}$.} \\[\baselineskip]
	Plugging this into GNU Octave,
	\begin{equation*}
		U^{-1} =
		\boxed{
			\frac{1}{8}
			\begin{pmatrix}
				3 & -1 \\
				-1 & 3
			\end{pmatrix}
		}
		.
	\end{equation*}
\end{enumerate}

\end{document}