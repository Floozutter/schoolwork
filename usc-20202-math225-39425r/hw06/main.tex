\documentclass[12pt]{article}
\usepackage[fleqn]{amsmath}
\usepackage{textcomp}
\usepackage{gensymb}
\usepackage{amsfonts}
\usepackage{enumitem}
%\usepackage{tikz}  % Include for figures.
%\usepackage{subfiles}  % Include for subfiles.

\newcommand{\HOMEWORKNUM}{6}
\newcommand{\NAME}{D. Choi}
\newcommand{\DATE}{2020-05-29}

\title{\vspace{-4\baselineskip}MATH 225 - Homework \#\HOMEWORKNUM}
\author{\NAME}
\date{\DATE}

\pagenumbering{gobble}  % Include for single-page document.

\begin{document}
\maketitle

\section*{1.}
\textit{Solve.}
\begin{align*}
	1x + 0y + 0z &= 0 \\
	0x + \frac{1}{2}y + \frac{1}{2}z &= 0 \\
	0x + \frac{1}{2}y + \frac{1}{2}z &= 1
\end{align*}
This system can be rewritten in matrix form $A\vec{v} = \vec{b}$ as
\begin{equation*}
	\begin{pmatrix}
		1 & 0 & 0 \\
		0 & \frac{1}{2} & \frac{1}{2} \\
		0 & \frac{1}{2} & \frac{1}{2}
	\end{pmatrix}
	\begin{pmatrix}
		x \\
		y \\
		z
	\end{pmatrix}
	=
	\begin{pmatrix}
		0 \\
		0 \\
		1
	\end{pmatrix}
	.
\end{equation*}
Such a matrix $A$ is an orthographic projection onto the plane $z = y$. \\
Since the vector $\vec{b}$ does not lie on the plane $z = y$,
\boxed{\text{no solutions exist}}
for $\vec{v}$.

\section*{2.}
\textit{Solve.}
\begin{align*}
	1x + 0y + 0z &= 10 \\
	0x + \frac{1}{2}y + \frac{1}{2}z &= 5 \\
	0x + \frac{1}{2}y + \frac{1}{2}z &= 5
\end{align*}
This system can be rewritten (again) in matrix form $A\vec{v} = \vec{c}$ as
\begin{equation*}
	\begin{pmatrix}
		1 & 0 & 0 \\
		0 & \frac{1}{2} & \frac{1}{2} \\
		0 & \frac{1}{2} & \frac{1}{2}
	\end{pmatrix}
	\begin{pmatrix}
		x \\
		y \\
		z
	\end{pmatrix}
	=
	\begin{pmatrix}
		10 \\
		5 \\
		5
	\end{pmatrix}
	.
\end{equation*}
Such a matrix $A$ is (still) an orthographic projection onto the plane
$z = y$. \\
Thus, solutions to the system are in the form
\begin{equation*}
	\vec{v} =
	\boxed{
		\vec{c} + s \begin{pmatrix} 0 \\ 1 \\ -1 \end{pmatrix}
	}
	,
\end{equation*}
where $s \in \mathbb{R}$.

\section*{3.}
\textit{Solve.}
\begin{equation*}
	\begin{pmatrix}
		\cos(30 \degree) & -\sin(30 \degree) & 0 \\
		\sin(30 \degree) & \cos(30 \degree) & 0 \\
		0 & 0 & 1
	\end{pmatrix}
	\begin{pmatrix}
		x \\
		y \\
		z
	\end{pmatrix}
	=
	\begin{pmatrix}
		\cos(40 \degree) \\
		\sin(40 \degree) \\
		10
	\end{pmatrix}
\end{equation*}
The matrix
\begin{equation*}
	\begin{pmatrix}
		\cos(30 \degree) & -\sin(30 \degree) & 0 \\
		\sin(30 \degree) & \cos(30 \degree) & 0 \\
		0 & 0 & 1
	\end{pmatrix}
\end{equation*}
is an instance of the rotation matrix about the z-axis
\begin{equation*}
	\text{Rot}_z(\theta) =
	\begin{pmatrix}
		\cos \theta & -\sin \theta & 0 \\
		\sin \theta & \cos \theta & 0 \\
		0 & 0 & 1
	\end{pmatrix}
\end{equation*}
where $\theta = 30 \degree$. \\
From this, the equation can be rewritten as
\begin{equation*}
	\text{Rot}_z(30 \degree)
	\begin{pmatrix} x \\ y \\ z \end{pmatrix}
	=
	\text{Rot}_z(40 \degree)
	\begin{pmatrix} 0 \\ 0 \\ 10 \end{pmatrix}
	.
\end{equation*}
Thus,
\begin{align*}
	\begin{pmatrix} x \\ y \\ z \end{pmatrix}
	&=
	(\text{Rot}_z(30 \degree))^{-1}
	\text{Rot}_z(40 \degree)
	\begin{pmatrix} 0 \\ 0 \\ 10 \end{pmatrix} \\
	&=
	\text{Rot}_z(-30 \degree)
	\text{Rot}_z(40 \degree)
	\begin{pmatrix} 0 \\ 0 \\ 10 \end{pmatrix} \\
	&=
	\text{Rot}_z(10 \degree)
	\begin{pmatrix} 0 \\ 0 \\ 10 \end{pmatrix} \\
	&=
	\boxed{
		\begin{pmatrix}
			\cos(10 \degree) \\
			\sin(10 \degree) \\
			10
		\end{pmatrix}
	}
	.
\end{align*}

\end{document}