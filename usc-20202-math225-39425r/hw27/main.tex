% Preamble.
\documentclass[12pt]{article}
\usepackage[margin=1.25in]{geometry}
\usepackage[fleqn]{amsmath}
\usepackage{textcomp}
\usepackage{gensymb}
\usepackage{amsfonts}
\usepackage{enumitem}
%\usepackage{tikz}  % Include for figures.
%\usepackage{subfiles}  % Include for subfiles.

%% Title macros.
\newcommand{\HOMEWORKNUM}{27}
\newcommand{\NAME}{D. Choi}
\newcommand{\DATE}{2020-07-02}

\title{\vspace{-2\baselineskip}MATH 225 - Homework \#\HOMEWORKNUM}
\author{\NAME}
\date{\DATE}

%% Formatting options.
%\pagenumbering{gobble}  % Include for single-page document.


% Document.
\begin{document}
\maketitle

\section*{1.}
\textit{Solve:}
\begin{equation*}
	y^{\prime\prime} - 9y = 0, \quad
	y(0) = 2, \quad
	y^\prime(0) = -1.
\end{equation*}
The roots of the characteristic equation are
\begin{equation*}
	r^2 - 9 = 0 \quad \Rightarrow \quad r_1 = 3, \quad r_2 = -3.
\end{equation*}
With $2$ distinct real roots, the general solution is
\begin{equation*}
	y(t) = c_1 e^{3 t} + c_2 e^{-3 t}.
\end{equation*}
Using the system of equations given by the initial conditions,
\begin{alignat*}{2}
	y(0) &=& \quad 2 &= c_1 + c_2 \\
	y^\prime(0) &=& \quad -1 &= 3 c_1 - 3 c_2,
\end{alignat*}
the coefficients can be solved for as
\begin{equation*}
	c_1 = \frac{5}{6}, \quad
	c_2 = \frac{7}{6}.
\end{equation*}
Thus,
\begin{equation*}
	\boxed{
		y(t) = \frac{5}{6} e^{3 t} + \frac{7}{6} e^{-3 t}
	}.
\end{equation*}

\section*{2.}
\textit{Solve:}
\begin{equation*}
	y^{\prime\prime} + 6y^\prime + 9y = 0, \quad
	y(0) = 1, \quad
	y(1) = 1.
\end{equation*}
The roots of the characteristic equation are
\begin{equation*}
	r^2 + 6r + 9 = 0 \quad \Rightarrow \quad r_1 = -3, \quad r_2 = -3.
\end{equation*}
With a repeated real root, the general solution is
\begin{equation*}
	y(t) = c_1 e^{-3 t} + c_2 e^{-3 t} t.
\end{equation*}
Using the system of equations given by the initial conditions,
\begin{alignat*}{2}
	y(0) &=& \quad 1 &= c_1 \\
	y(1) &=& \quad 1 &= c_1 e^{-3} +  c_2 e^{-3},
\end{alignat*}
the coefficients can be solved for as
\begin{equation*}
	c_1 = 1, \quad
	c_2 = e^3 - 1.
\end{equation*}
Thus,
\begin{equation*}
	\boxed{
		y(t) = e^{-3 t} + (e^3 - 1) e^{-3 t} t
	}.
\end{equation*}

\section*{3.}
\textit{Solve:}
\begin{equation*}
	y^{\prime\prime} + 7y^\prime + 10y = 0, \quad
	y(0) = -1, \quad
	y^\prime(0) = 0.
\end{equation*}
The roots of the characteristic equation are
\begin{equation*}
	r^2 + 7r + 10 = 0 \quad \Rightarrow \quad r_1 = -2, \quad r_2 = -5.
\end{equation*}
With $2$ distinct real roots, the general solution is
\begin{equation*}
	y(t) = c_1 e^{-2 t} + c_2 e^{-5 t}.
\end{equation*}
Using the system of equations given by the initial conditions,
\begin{alignat*}{2}
	y(0) &=& \quad -1 &= c_1 + c_2 \\
	y^\prime(0) &=& \quad 0 &= -2 c_1 - 5 c_2,
\end{alignat*}
the coefficients can be solved for as
\begin{equation*}
	c_1 = -\frac{5}{3}, \quad
	c_2 = \frac{2}{3}.
\end{equation*}
Thus,
\begin{equation*}
	\boxed{
		y(t) = -\frac{5}{3} e^{-2 t} + \frac{2}{3} e^{-5 t}
	}.
\end{equation*}

\end{document}