% Preamble.
\documentclass[12pt]{article}
\usepackage[margin=1.25in]{geometry}
\usepackage[fleqn]{amsmath}
\usepackage{textcomp}
\usepackage{gensymb}
\usepackage{amsfonts}
\usepackage{enumitem}
%\usepackage{tikz}  % Include for figures.
%\usepackage{subfiles}  % Include for subfiles.

\title{\vspace{-2\baselineskip}MATH 225 - Midterm}
\author{D. Choi}
\date{2020-06-12}

%% Formatting options.
%\pagenumbering{gobble}  % Include for single-page document.

% Norm macro.
\newcommand{\norm}[1]{\left\lVert#1\right\rVert}
% Augmented matrix environment.
% https://tex.stackexchange.com/a/2238
\newenvironment{amatrix}[1]{%
  \left(\begin{array}{@{}*{#1}{c}|c@{}}
}{%
  \end{array}\right)
}


% Document.
\begin{document}
\maketitle

\section*{1.}
\textit{Let $A$ be the $3 \times 3$ orthographic projection matrix that
projects vectors onto the plane $x + 2y + 3z = 0$. Find $A$.}
\\[\baselineskip]
The equation for the plane $x + 2y + 3z = 0$ can be rewritten as
\begin{equation*}
	\begin{pmatrix} 1 \\ 2 \\ 3 \end{pmatrix}
	\cdot
	\begin{pmatrix} x \\ y \\ z \end{pmatrix}
	= \vec{n} \cdot \vec{v}
	= 0
	.
\end{equation*}
In this form, it is apparent that $\vec{n}$ is a normal vector to the plane.
\\
For any vector $\vec{v}$, the vector rejection of $\vec{v}$ from $\vec{n}$
(which is $\vec{v} - \text{proj}_{\vec{n}}\vec{v}$) will be the orthographic
projection of $\vec{v}$ onto the plane. Thus, for the standard basis unit
vectors,
\begin{gather*}
	\begin{pmatrix} 1 \\ 0 \\ 0 \end{pmatrix}
	\overset{A}{\longrightarrow}
	\begin{pmatrix} 1 \\ 0 \\ 0 \end{pmatrix}
	-
	\frac{
		\begin{pmatrix} 1 \\ 0 \\ 0 \end{pmatrix}
		\cdot
		\begin{pmatrix} 1 \\ 2 \\ 3 \end{pmatrix}
	}{
		\norm{\begin{pmatrix} 1 \\ 2 \\ 3 \end{pmatrix}}^2
	}
	\begin{pmatrix} 1 \\ 2 \\ 3 \end{pmatrix}
	=
	\begin{pmatrix} 1 \\ 0 \\ 0 \end{pmatrix}
	-
	\frac{1}{14}
	\begin{pmatrix} 1 \\ 2 \\ 3 \end{pmatrix}
	=
	\begin{pmatrix}
		\frac{13}{14} \\ -\frac{2}{14} \\ -\frac{3}{14}
	\end{pmatrix}
	\\
	\begin{pmatrix} 0 \\ 1 \\ 0 \end{pmatrix}
	\overset{A}{\longrightarrow}
	\ldots
	\\
	\begin{pmatrix} 0 \\ 0 \\ 1 \end{pmatrix}
	\overset{A}{\longrightarrow}
	\ldots
\end{gather*}
And so,
\begin{equation*}
	\boxed{
		A =
		\begin{pmatrix}
			\frac{13}{14} & ? & ? \\
			-\frac{2}{14} & ? & ? \\
			-\frac{3}{14} & ? & ?
		\end{pmatrix}
	}
	.
\end{equation*}

\end{document}