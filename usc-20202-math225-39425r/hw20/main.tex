% Preamble.
\documentclass[12pt]{article}
\usepackage[margin=1.25in]{geometry}
\usepackage[fleqn]{amsmath}
\usepackage{textcomp}
\usepackage{gensymb}
\usepackage{amsfonts}
\usepackage{enumitem}
%\usepackage{tikz}  % Include for figures.
%\usepackage{subfiles}  % Include for subfiles.

%% Title macros.
\newcommand{\HOMEWORKNUM}{20}
\newcommand{\NAME}{D. Choi}
\newcommand{\DATE}{2020-06-23}

\title{\vspace{-2\baselineskip}MATH 225 - Homework \#\HOMEWORKNUM}
\author{\NAME}
\date{\DATE}

%% Formatting options.
%\pagenumbering{gobble}  % Include for single-page document.


% Document.
\begin{document}
\maketitle

\section*{1.}
\textit{Suppose that $A$ and its reduced row echelon form are}
\begin{equation*}
	A
	=
	\begin{pmatrix}
		0 & 2 & 0 & -4 & 0 & 6 \\
		0 & -4 & -1 & 7 & 0 & -16 \\
		0 & 6 & 0 & -12 & 3 & 15 \\
		0 & 4 & -1 & -9 & 0 & 8
	\end{pmatrix}
	\sim
	\begin{pmatrix}
		0 & 1 & 0 & -2 & 0 & 3 \\
		0 & 0 & 1 & 1 & 0 & 4 \\
		0 & 0 & 0 & 0 & 1 & -1 \\
		0 & 0 & 0 & 0 & 0 & 0
	\end{pmatrix}
\end{equation*}
\begin{enumerate}[label=(\alph*)]
	\item \textit{The null space $\text{nul}(A)$ is a subspace of
	$\mathbb{R}^p$ for what $p$? The column space $\text{col}(A)$ is a
	subspace of $\mathbb{R}^p$ for what $p$?} \\[\baselineskip]
	The null space (kernel) is a subspace of the domain,
	$\boxed{\mathbb{R}^6}$, where $6$ is the number of columns of $A$. \\
	The column space (image) is a subspace of the codomain,
	$\boxed{\mathbb{R}^4}$, where $4$ is the number of rows of $A$.
	
	\item \textit{What are the dimensions $\text{dim}(\text{nul}(A))$ and
	$\text{dim}(\text{col}(A))$?} \\[\baselineskip]
	$A$ has a nullity of $\boxed{3}$, because its RREF has $3$ free variables.
	\\
	$A$ has a rank of $\boxed{3}$, because its RREF has $3$ leading variables
	(nonzero rows). \\
	This result can be verified using the rank-nullity theorem:
	\begin{align*}
		\text{dim}(\text{col}(A)) + \text{dim}(\text{nul}(A))
		&=
		\text{dim}(\text{domain}(A))
		\\
		\text{rank}(A) + \text{nullity}(A)
		&=
		\text{number of columns of } A
		\\
		3 + 3
		&=
		6
	\end{align*}
	
	\item \textit{Find a basis for the column space $\text{col}(A)$.}
	\\[\baselineskip]
	The columns of $A$ that correspond to the columns that contain leading
	variables in $\text{rref}(A)$,
	\begin{equation*}
		\boxed{
			\left\{
				\begin{pmatrix} 2 \\ -4 \\ 6 \\ 4 \end{pmatrix},
				\begin{pmatrix} 0 \\ -1 \\ 0 \\ -1 \end{pmatrix},
				\begin{pmatrix} 0 \\ 0 \\ 3 \\ 0 \end{pmatrix}
			\right\}
		}
		,
	\end{equation*}
	form a basis for the image of $A$. \\[\baselineskip]
	Alternatively, the nonzero rows of $\text{rref}(A^\text{T})$ also form a
	basis for the image of $A$.
	Because
	\begin{equation*}
		\text{rref}(A^\text{T})
		=
		\begin{pmatrix}
			1 & 0 & 0 & 4 \\
			0 & 1 & 0 & 1 \\
			0 & 0 & 1 & 0 \\
			0 & 0 & 0 & 0 \\
			0 & 0 & 0 & 0 \\
			0 & 0 & 0 & 0
		\end{pmatrix}
		,
	\end{equation*}
	another valid basis is
	\begin{equation*}
		\boxed{
			\left\{
				\begin{pmatrix} 1 \\ 0 \\ 0 \\ 4 \end{pmatrix},
				\begin{pmatrix} 0 \\ 1 \\ 0 \\ 1 \end{pmatrix},
				\begin{pmatrix} 0 \\ 0 \\ 1 \\ 0 \end{pmatrix}
			\right\}
		}
		.
	\end{equation*}
	This basis can be shown to span the same subspace as the first basis by
	row reducing the matrix with rows that are the first basis's vectors:
	\begin{equation*}
		\text{rref}
			\begin{pmatrix}
				2 & -4 & 6 & 4 \\
				0 & -1 & 0 & -1 \\
				0 & 0 & 3 & 0
			\end{pmatrix}
		=
		\begin{pmatrix}
			1 & 0 & 0 & 4 \\
			0 & 1 & 0 & 1 \\
			0 & 0 & 1 & 0 \\
		\end{pmatrix}
		.
	\end{equation*}
\end{enumerate}

\end{document}
