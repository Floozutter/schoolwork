% Preamble.
\documentclass[12pt]{article}
\usepackage[margin=1.25in]{geometry}
\usepackage[fleqn]{amsmath}
\usepackage{textcomp}
\usepackage{gensymb}
\usepackage{amsfonts}
\usepackage{enumitem}
%\usepackage{tikz}  % Include for figures.
%\usepackage{subfiles}  % Include for subfiles.

%% Title macros.
\newcommand{\HOMEWORKNUM}{21}
\newcommand{\NAME}{D. Choi}
\newcommand{\DATE}{2020-06-24}

\title{\vspace{-2\baselineskip}MATH 225 - Homework \#\HOMEWORKNUM}
\author{\NAME}
\date{\DATE}

%% Formatting options.
%\pagenumbering{gobble}  % Include for single-page document.

% Macros.
%% Contextualized by input/output bases.
\newcommand{\based}[3]{{\{#1\}}_{#2}^{#3}}


% Document.
\begin{document}
\maketitle

\section*{1.}
\begin{equation*}
	A = \begin{pmatrix} 5 & 1 \\ 1 & 5 \end{pmatrix}
\end{equation*}
\begin{enumerate}[label=(\alph*)]
	\item \textit{Let}
	\begin{equation*}
		\vec{a} = \begin{pmatrix} 1 \\ 1 \end{pmatrix}, \quad
		\vec{b} = \begin{pmatrix} 2 \\ 2 \end{pmatrix}, \quad
		\vec{c} = \begin{pmatrix} -1 \\ 1 \end{pmatrix},\quad
		\vec{d} = \begin{pmatrix} 5 \\ -5 \end{pmatrix},\quad
		\vec{e} = \begin{pmatrix} 10 \\ 0 \end{pmatrix}.
	\end{equation*}
	\textit{Which are eigenvectors of $A$? Find their corresponding
	eigenvalues.} \\[\baselineskip]
	For a square matrix $A$, a vector $\vec{v}$ is an eigenvector of $A$ if
	there is a scalar $\lambda$ such that $A\vec{v} = \lambda \vec{v}$.
	\\[\baselineskip]
	For $\vec{a} = \begin{pmatrix} 1 \\ 1 \end{pmatrix}$,
	\begin{equation*}
		A \vec{a}
		=
		\begin{pmatrix} 5 & 1 \\ 1 & 5 \end{pmatrix}
		\begin{pmatrix} 1 \\ 1 \end{pmatrix}
		=
		\begin{pmatrix} 6 \\ 6 \end{pmatrix}
		=
		6
		\begin{pmatrix} 1 \\ 1 \end{pmatrix}
		=
		6 \vec{a}
		.
	\end{equation*}
	Thus, $\vec{a}$ is an eigenvector of $A$ with a corresponding eigenvalue
	of \boxed{6}.
	\\[\baselineskip]
	For $\vec{b} = \begin{pmatrix} 2 \\ 2 \end{pmatrix}$,
	\begin{equation*}
		\vec{b} = 2 \vec{a}
		.
	\end{equation*}
	Because $\vec{b}$ is a scalar multiple of $\vec{a}$, an eigenvector of
	$A$, $\vec{b}$ is also an eigenvector of $A$. Furthermore, $\vec{b}$
	shares $\vec{a}$'s corresponding eigenvalue of \boxed{6}.
	\\[\baselineskip]
	For $\vec{c} = \begin{pmatrix} -1 \\ 1 \end{pmatrix}$,
	\begin{equation*}
		A \vec{c}
		=
		\begin{pmatrix} 5 & 1 \\ 1 & 5 \end{pmatrix}
		\begin{pmatrix} -1 \\ 1 \end{pmatrix}
		=
		\begin{pmatrix} -4 \\ 4 \end{pmatrix}
		=
		4
		\begin{pmatrix} -1 \\ 1 \end{pmatrix}
		=
		4 \vec{c}
		.
	\end{equation*}
	Thus, $\vec{c}$ is an eigenvector of $A$ with a corresponding eigenvalue
	of \boxed{4}.
	\\[\baselineskip]
	For $\vec{d} = \begin{pmatrix} 5 \\ -5 \end{pmatrix}$,
	\begin{equation*}
		\vec{d} = -5 \vec{c}
		.
	\end{equation*}
	Because $\vec{d}$ is a scalar multiple of $\vec{c}$, an eigenvector of
	$A$, $\vec{d}$ is also an eigenvector of $A$. Furthermore, $\vec{d}$
	shares $\vec{c}$'s corresponding eigenvalue of \boxed{4}.
	\\[\baselineskip]
	For $\vec{e} = \begin{pmatrix} 10 \\ 0 \end{pmatrix}$,
	\begin{equation*}
		A \vec{e}
		=
		\begin{pmatrix} 5 & 1 \\ 1 & 5 \end{pmatrix}
		\begin{pmatrix} 10 \\ 0 \end{pmatrix}
		=
		\begin{pmatrix} 50 \\ 10 \end{pmatrix}
		\neq
		\lambda
		\begin{pmatrix} 10 \\ 0 \end{pmatrix}
		\text{ for any $\lambda$.}
	\end{equation*}
	Thus, $\vec{e}$ is not an eigenvector of $A$.
	
	\item \textit{Let $\mathcal{S}$ be the standard basis, and let}
	\begin{equation*}
		\based{T}{\mathcal{S}}{\mathcal{S}} = A.
	\end{equation*}
	\textit{Find a basis $\mathcal{B}$ made of the eigenvectors of $A$, and
	find $\based{T}{\mathcal{B}}{\mathcal{B}}$.} \\[\baselineskip]
	Let
	\begin{equation*}
		\mathcal{B}
		=
		\left\{
			\begin{pmatrix} 1 \\ 1 \end{pmatrix},
			\begin{pmatrix} -1 \\ 1 \end{pmatrix}
		\right\}
		.
	\end{equation*}
	Using $\mathcal{B}$,
	\begin{align*}
		\based{T}{\mathcal{B}}{\mathcal{B}}
		&=
		\based{T}{\mathcal{S}}{\mathcal{B}}
		\based{T}{\mathcal{S}}{\mathcal{S}}
		\based{I}{\mathcal{B}}{\mathcal{S}}
		\\
		&=
		\begin{pmatrix}
			\frac{1}{2} & \frac{1}{2} \\
			-\frac{1}{2} & \frac{1}{2}
		\end{pmatrix}
		\begin{pmatrix} 5 & 1 \\ 1 & 5 \end{pmatrix}
		\begin{pmatrix} 1 & -1 \\ 1 & 1 \end{pmatrix}
		\\
		&=
		\boxed{
			\begin{pmatrix} 6 & 0 \\ 0 & 4 \end{pmatrix}
		}
		.
	\end{align*}
	Note the diagonal elements of $\based{T}{\mathcal{B}}{\mathcal{B}}$ are
	the eigenvalues of $A$.
\end{enumerate}
\newpage

\section*{2}
\textit{Suppose that $A$ is a $3 \times 4$ matrix.}
\begin{enumerate}[label=(\alph*)]
	\item \textit{Is it possible that $\text{dim}(\text{nul}(A)) = 0$?}
	\\[\baselineskip]
	\boxed{\text{No.}} By the rank-nullity theorem,
	\begin{equation*}
		\text{rank}(A) + \text{nullity}(A) = \text{dim}(\text{domain}(A))
		.
	\end{equation*}
	The dimension of the domain of $A$ is $4$, because $A$ has $4$ columns. \\
	The rank of $A$ is at maximum 3, because $A$ is a $3 \times 4$ matrix. \\
	As a result of this, the nullity of $A$ must be at least $1$.
	
	\item \textit{If $\text{dim}(\text{nul}(A)) = 1$, what can you say about
	$\text{col}(A)$?} \\[\baselineskip]
	\boxed{\text{dim}(\text{col}(A)) = \text{rank}(A) = 4 - 1 = 3.}

	\item \textit{If $\text{dim}(\text{nul}(A)) = 2$, what can you say about
	$\text{col}(A)$?} \\[\baselineskip]
	\boxed{\text{dim}(\text{col}(A)) = \text{rank}(A) = 4 - 2 = 2.}
	
	\item \textit{If $\text{dim}(\text{nul}(A)) = 3$, what can you say about
	$\text{col}(A)$?} \\[\baselineskip]
	\boxed{\text{dim}(\text{col}(A)) = \text{rank}(A) = 4 - 3 = 1.}
	
	\item \textit{If $\text{dim}(\text{nul}(A)) = 4$, what can you say about
	$\text{col}(A)$?} \\[\baselineskip]
	\boxed{\text{dim}(\text{col}(A)) = \text{rank}(A) = 4 - 4 = 0.}
\end{enumerate}
\newpage

\section*{3}
\textit{Let}
\begin{equation*}
	A = \begin{pmatrix} 8 & -10 \\ 5 & -7 \end{pmatrix},
	\qquad
	\vec{v}_1 = \begin{pmatrix} 2 \\ 1 \end{pmatrix}, \quad
	\vec{v}_2 = \begin{pmatrix} 1 \\ 1 \end{pmatrix}.
\end{equation*}
\begin{enumerate}[label=(\alph*)]
	\item \textit{Show that the vectors $\vec{v}_1$ and $\vec{v}_2$ are
	eigenvectors of $A$ and find their associated eigenvalues.}
	\begin{gather*}
		A \vec{v}_1
		=
		\begin{pmatrix} 8 & -10 \\ 5 & -7 \end{pmatrix}
		\begin{pmatrix} 2 \\ 1 \end{pmatrix}
		=
		\begin{pmatrix} 6 \\ 3 \end{pmatrix}
		=
		3 \begin{pmatrix} 2 \\ 1 \end{pmatrix}
		=
		3 \vec{v}_1
		, \\
		A \vec{v}_2
		=
		\begin{pmatrix} 8 & -10 \\ 5 & -7 \end{pmatrix}
		\begin{pmatrix} 1 \\ 1 \end{pmatrix}
		=
		\begin{pmatrix} -2 \\ -2 \end{pmatrix}
		=
		-2 \begin{pmatrix} 1 \\ 1 \end{pmatrix}
		=
		-2 \vec{v}_2
		.
	\end{gather*}
	Thus, $\boxed{\lambda_1 = 3}, \boxed{\lambda_2 = -2}$.
	
	\item \textit{Express the vector}
	\begin{equation*}
		\vec{x} = \begin{pmatrix} -4 \\ -1 \end{pmatrix}
	\end{equation*}
	\textit{as a linear combination of $\vec{v}_1$ and $\vec{v}_2$.}
	\\[\baselineskip]
	Solving
	\begin{equation*}
		\begin{pmatrix} 2 & 1 \\ 1 & 1 \end{pmatrix}
		\begin{pmatrix} a \\ b \end{pmatrix}
		=
		\begin{pmatrix} -4 \\ -1 \end{pmatrix}
		,
	\end{equation*}
	$a = -3$, and $b = 2$. \\
	Thus,
	\begin{equation*}
		\boxed{
			\vec{x}
			=
			-3 \vec{v}_1 + 2 \vec{v}_2
		}
		.
	\end{equation*}
\end{enumerate}
\newpage

\section{Extra.}
\textit{Consider the Fibonacci sequence and the following:}
\begin{equation*}
	\begin{pmatrix} 0 & 1 \\ 1 & 1 \end{pmatrix}
	\begin{pmatrix} 2 \\ 3 \end{pmatrix}
	=
	\begin{pmatrix} 3 \\ 5 \end{pmatrix}
	, \quad
	\begin{pmatrix} 0 & 1 \\ 1 & 1 \end{pmatrix}
	\begin{pmatrix} 3 \\ 5 \end{pmatrix}
	=
	\begin{pmatrix} 5 \\ 8 \end{pmatrix}
	.
\end{equation*}
\textit{Find the 1000th Fibonacci number.} \\[\baselineskip]
Let $0$ be the $0$th Fibonacci number.
\begin{equation*}
	\begin{pmatrix} 0 & 1 \\ 1 & 1 \end{pmatrix}^n
	\begin{pmatrix} 0 \\ 1 \end{pmatrix}
	=
	\begin{pmatrix}
		n\text{th Fibonacci number} \\
		(n+1)\text{th Fibonacci number}
	\end{pmatrix}
	.
\end{equation*}
Thus, let
\begin{equation*}
	\begin{pmatrix} 0 & 1 \\ 1 & 1 \end{pmatrix}^n
	\begin{pmatrix} 0 \\ 1 \end{pmatrix}
	=
	\begin{pmatrix}
		a \\
		\_
	\end{pmatrix}
	.
\end{equation*}
\boxed{a} is the 1000th Fibonacci number.


\end{document}