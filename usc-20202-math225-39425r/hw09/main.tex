% Preamble.
\documentclass[12pt]{article}
\usepackage[margin=1.25in]{geometry}
\usepackage[fleqn]{amsmath}
\usepackage{textcomp}
\usepackage{gensymb}
\usepackage{amsfonts}
\usepackage{enumitem}
%\usepackage{tikz}  % Include for figures.
%\usepackage{subfiles}  % Include for subfiles.

%% Title macros.
\newcommand{\HOMEWORKNUM}{9}
\newcommand{\NAME}{D. Choi}
\newcommand{\DATE}{2020-06-03}

\title{\vspace{-2\baselineskip}MATH 225 - Homework \#\HOMEWORKNUM}
\author{\NAME}
\date{\DATE}

%% Formatting options.
%\pagenumbering{gobble}  % Include for single-page document.


% Document.
\begin{document}
\maketitle

\section*{1.}
\textit{Find the $2 \times 2$ projection matrix that projects vectors
orthogonally onto $y = -5x$.} \\[\baselineskip]
Let $\phi = \tan^{-1}(-5)$, so that $y = \tan(\phi)x$. \\
Let $\hat u(\theta)$ be the unit vector at an angle of $\theta$ to the
positive x-axis:
\begin{equation*}
	\hat u(\theta) = \begin{pmatrix} \cos \theta \\ \sin \theta \end{pmatrix}
	.
\end{equation*}
Let $M$ be the orthographic projection matrix onto $y = \tan(\phi)x$ with
elements $a$, $b$, $c$, and $d$:
\begin{equation*}
	M = \begin{pmatrix} a & b \\ c & d \end{pmatrix}
	.
\end{equation*}
The vectors $(a, c)$ and $(b, d)$ can be found by computing geometrically
where the standard unit basis vectors $(1, 0)$ and $(0, 1)$ respectively will
land upon being transformed by $M$. $M$ can be considered as the vector
projection onto $\hat u(\phi)$:
\footnotesize
\begin{gather*}
	\begin{pmatrix} a \\ c \end{pmatrix}
	=
	M \begin{pmatrix} 1 \\ 0 \end{pmatrix}
	=
	\text{proj}_{\hat u(\phi)} \hat u(0)
	=
	\cos(\phi) \hat u(\phi)
	=
	\cos(\phi)
	\begin{pmatrix}
		\cos(\phi) \\
		\sin(\phi)
	\end{pmatrix}
	=
	\begin{pmatrix}
		\cos^2(\phi) \\
		\frac{1}{2}\sin(2\phi)
	\end{pmatrix}
	\\
	\begin{pmatrix} b \\ d \end{pmatrix}
	=
	M \begin{pmatrix} 0 \\ 1 \end{pmatrix}
	=
	\text{proj}_{\hat u(\phi)} \hat u(\frac{\pi}{2})
	=
	\cos(\phi - \frac{\pi}{2}) \hat u(\phi)
	=
	\sin(\phi)
	\begin{pmatrix}
		\cos(\phi) \\
		\sin(\phi)
	\end{pmatrix}
	=
	\begin{pmatrix}
		\frac{1}{2}\sin(2\phi) \\
		\sin^2(\phi)
	\end{pmatrix}
	.
\end{gather*}
Note that 
$\cos(\theta - \frac{\pi}{2})$ = $\sin(\theta)$ and	
$\sin(\theta)\cos(\theta) = \frac{1}{2}\sin(2\theta)$.
\normalsize
\\[\baselineskip]
Thus,
\begin{equation*}
	M
	=
	\begin{pmatrix}
		\cos^2(\phi) & \frac{1}{2}\sin(2\phi) \\
		\frac{1}{2}\sin(2\phi) & \sin^2(\phi)
	\end{pmatrix}
	=
	\boxed{
		\begin{pmatrix}
			\frac{1}{26} & -\frac{5}{26} \\
			-\frac{5}{26} & \frac{25}{26}
		\end{pmatrix}
	}
	.
\end{equation*}
To verify,
\begin{gather*}
	\begin{pmatrix}
		\frac{1}{26} & -\frac{5}{26} \\
		-\frac{5}{26} & \frac{25}{26}
	\end{pmatrix}
	\begin{pmatrix} 1 \\ -5 \end{pmatrix}
	=
	\begin{pmatrix} 1 \\ -5 \end{pmatrix}
	\\
	\begin{pmatrix}
		\frac{1}{26} & -\frac{5}{26} \\
		-\frac{5}{26} & \frac{25}{26}
	\end{pmatrix}
	\begin{pmatrix} 5 \\ 1 \end{pmatrix}
	=
	\begin{pmatrix} 0 \\ 0 \end{pmatrix}
	.
\end{gather*}

\section*{2.}
\textit{Calculate the dot product of $c = (-4, -9)$ and $d = (-1, 2)$. Do the
vectors form an acute angle, right angle, or obtuse angle?}\\[\baselineskip]
Note that algebraically $a \cdot b = a^T b$, and that geometrically
$a \cdot b =
\left\lVert a \right\rVert \left\lVert b \right\rVert \cos(\theta)$
where $\theta$ is the angle between
$a$ and $b$.
\\[\baselineskip]
The dot product can be computed algebraically:
\begin{equation*}
	c \cdot d = c^T d
	=
	\begin{pmatrix} -4 & -9 \end{pmatrix}
	\begin{pmatrix} -1 \\ 2 \end{pmatrix}
	=
	\boxed{-14}
	.
\end{equation*}
Because $c \cdot d$ is negative while
$c \cdot d =
\left\lVert c \right\rVert \left\lVert d \right\rVert \cos(\theta)$, \\
\boxed{\text{the vectors form an obtuse angle}}.

\section*{3.}
\textit{If $a = (6, -1, 3)$, for what value of $c$ is the vector
$b = (4, c, -2)$ perpendicular to $a$?}\\[\baselineskip]
By the geometric definition of the dot product, the dot product of two
perpendicular vectors must equal $0$. \\[\baselineskip]
Thus,
\begin{equation*}
	c \cdot d = c^T d =
	\begin{pmatrix} 6 & -1 & 3 \end{pmatrix}
	\begin{pmatrix} 4 \\ c \\ -2 \end{pmatrix}
	= 18 - c = 0.
\end{equation*}
Solving for $c$, \boxed{$\textit{c} = 18$}.

\section*{4.}
\textit{Find the angle between the vector $(1, 1, 1)$ and the vector
$(-1, -2, 0)$. Explain before you find it whether you think the answer should
be positive, negative, or $0$.} \\[\baselineskip]
Because the vectors are in approximately opposite directions (note that
``approximately opposite'' is very rigorous), the angle should be obtuse.
\\[\baselineskip]
\begin{equation*}
	(1, 1, 1) \cdot (-1, -2, 0)
	=
	-3
	=
	\sqrt{15} \cos(\theta)
\end{equation*}
Solving for $\theta$,
$\boxed{\theta = \cos^{-1}(-\sqrt{\frac{3}{5}}) \approx 140.8 \degree}$.

\end{document}