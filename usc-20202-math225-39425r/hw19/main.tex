% Preamble.
\documentclass[12pt]{article}
\usepackage[margin=1.25in]{geometry}
\usepackage[fleqn]{amsmath}
\usepackage{textcomp}
\usepackage{gensymb}
\usepackage{amsfonts}
\usepackage{enumitem}
%\usepackage{tikz}  % Include for figures.
%\usepackage{subfiles}  % Include for subfiles.

%% Title macros.
\newcommand{\HOMEWORKNUM}{19}
\newcommand{\NAME}{D. Choi}
\newcommand{\DATE}{2020-06-22}

\title{\vspace{-2\baselineskip}MATH 225 - Homework \#\HOMEWORKNUM}
\author{\NAME}
\date{\DATE}

%% Formatting options.
%\pagenumbering{gobble}  % Include for single-page document.

% Macros.
%% Contextualized by input/output bases.
\newcommand{\based}[3]{{\{#1\}}_{#2}^{#3}}


% Document.
\begin{document}
\maketitle

\section*{1.}
\textit{Let $\mathcal{S}$ be the standard basis, and let}
\begin{gather*}
	\mathcal{A}
	=
	\left\{
		\begin{pmatrix} -1 \\ 0 \end{pmatrix},
		\begin{pmatrix} 0 \\ -1 \end{pmatrix}
	\right\}
	,
	\\
	\mathcal{B}
	=
	\left\{
		\begin{pmatrix} -1 \\ 1 \end{pmatrix},
		\begin{pmatrix} 1 \\ 1 \end{pmatrix}
	\right\}
	.
\end{gather*}
\textit{Also, let}
\begin{equation*}
	\based{T}{\mathcal{S}}{\mathcal{S}}
	=
	\begin{pmatrix}
		-1 & 3 \\
		3 & -1
	\end{pmatrix}
	.
\end{equation*}
\begin{enumerate}[label=(\alph*)]
	\item \textit{Find $\based{T}{\mathcal{B}}{\mathcal{A}}$.}
	\\[\baselineskip]
	Let $\vec{w}_1$ and $\vec{w}_2$ be the vectors of $\mathcal{B}$ so that
	\begin{equation*}
		\vec{w}_1 = \begin{pmatrix} -1 \\ 1 \end{pmatrix},
		\vec{w}_1 = \begin{pmatrix} 1 \\ 1 \end{pmatrix}.
	\end{equation*}
	Using $\based{I}{\mathcal{B}}{\mathcal{S}}$ (the change of basis matrix
	from $\mathcal{B}$ to standard basis) and
	$\based{I}{\mathcal{S}}{\mathcal{A}}$ (the change of basis matrix from
	standard basis to $\mathcal{A}$), $\based{T}{\mathcal{B}}{\mathcal{A}}$
	can be expressed as
	\begin{align*}
		\based{T}{\mathcal{B}}{\mathcal{A}}
		&=
		\based{I}{\mathcal{S}}{\mathcal{A}}
		\based{T}{\mathcal{S}}{\mathcal{S}}
		\based{I}{\mathcal{B}}{\mathcal{S}}
		\\
		&=
		\begin{pmatrix}
			\vert & \vert \\
			\based{I}{\mathcal{S}}{\mathcal{A}}
			\based{T}{\mathcal{S}}{\mathcal{S}}
			\vec{w}_1 &
			\based{I}{\mathcal{S}}{\mathcal{A}}
			\based{T}{\mathcal{S}}{\mathcal{S}}
			\vec{w}_2 \\
			\vert & \vert
		\end{pmatrix}
	\end{align*}
	Or, more concisely,
	\begin{equation*}
		\based{T}{\mathcal{B}}{\mathcal{A}}
		=
		\begin{pmatrix}
			\vert & \vert \\
			{\{T \vec{w}_1\}}_{\mathcal{A}} &
			{\{T \vec{w}_2\}}_{\mathcal{A}} \\
			\vert & \vert
		\end{pmatrix}
		.
	\end{equation*}
	
	\item \textit{Find $\based{T}{\mathcal{B}}{\mathcal{B}}$.}
	\begin{equation*}
		\based{T}{\mathcal{B}}{\mathcal{B}}
		=
		\begin{pmatrix}
			\vert & \vert \\
			{\{T \vec{w}_1\}}_{\mathcal{B}} &
			{\{T \vec{w}_2\}}_{\mathcal{B}} \\
			\vert & \vert
		\end{pmatrix}
		.
	\end{equation*}
\end{enumerate}

\section*{2.}
\begin{equation*}
	\vec{a} = \begin{pmatrix} -4 \\ -3 \\ 0 \end{pmatrix},
	\vec{b} = \begin{pmatrix} 3 \\ 4 \\ 1 \end{pmatrix}
\end{equation*}
\textit{The vectors $\vec{a}$ and $\vec{b}$ are orthogonal. Find a vector
$\vec{c}$ that is orthogonal to both of these vectors.} \\[\baselineskip]
The vector that is the cross product of two other vectors is, by definition,
orthogonal to both of the two other vectors. \\
Thus, let $\vec{c}$ be the cross product of $\vec{a}$ and $\vec{b}$:
\begin{equation*}
	\vec{c}
	=
	\vec{a} \times \vec{b}
	=
	\begin{pmatrix} -4 \\ -3 \\ 0 \end{pmatrix}
	\times
	\begin{pmatrix} 3 \\ 4 \\ 1 \end{pmatrix}
	=
	\boxed{
		\begin{pmatrix} -3 \\ 4 \\ -7 \end{pmatrix}
	}
	.
\end{equation*}
The vector $\vec{c}$'s orthogonality to both $\vec{a}$ and $\vec{b}$ can be
verified by confirming that
\begin{gather*}
	\vec{c} \cdot \vec{a} = 0, \\
	\vec{c} \cdot \vec{b} = 0.
\end{gather*}
Any other vector that fulfills the criteria can be found by computing
\begin{equation*}
	s \cdot \vec{c}
\end{equation*}
with an arbitrary scalar $s$.
\end{document}
