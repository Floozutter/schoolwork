\documentclass[12pt]{article}
\usepackage[margin=1.25in]{geometry}
\usepackage[fleqn]{amsmath}
\usepackage{textcomp}
\usepackage{gensymb}
\usepackage{amsfonts}
\usepackage{enumitem}
%\usepackage{tikz}  % Include for figures.
%\usepackage{subfiles}  % Include for subfiles.

\newcommand{\HOMEWORKNUM}{8}
\newcommand{\NAME}{D. Choi}
\newcommand{\DATE}{2020-06-02}

\title{\vspace{-2\baselineskip}MATH 225 - Homework \#\HOMEWORKNUM}
\author{\NAME}
\date{\DATE}

%\pagenumbering{gobble}  % Include for single-page document.

% Augmented matrix environment.
% https://tex.stackexchange.com/a/2238
\newenvironment{amatrix}[1]{%
  \left(\begin{array}{@{}*{#1}{c}|c@{}}
}{%
  \end{array}\right)
}

\begin{document}
\maketitle

\section*{1.}
\textit{Solve.}
\begin{align*}
	x + 2y + 3z + 5w &= 8 \\
	2x + 4y + 6z + 10w &= 16 \\
	x + y + z + w &= 10
\end{align*}
This system can be rewritten in matrix form $A\vec{v} = \vec{b}$ as
\begin{equation*}
	\begin{pmatrix}
		1 & 2 & 3 & 5 \\
		2 & 4 & 6 & 10 \\
		1 & 1 & 1 & 1
	\end{pmatrix}
	\begin{pmatrix} x \\ y \\ z \\ w \end{pmatrix}
	=
	\begin{pmatrix} 8 \\ 16 \\ 10 \end{pmatrix}
\end{equation*}
The coefficient matrix $A$ and the target vector $\vec{b}$ can be concatenated
column-wise to create the augmented matrix $(A|\vec{b})$.
\begin{equation*}
	(A|\vec{b}) =
	\begin{amatrix}{4}
		1 & 2 & 3 & 5 & 8 \\
		2 & 4 & 6 & 10 & 16 \\
		1 & 1 & 1 & 1 & 10
	\end{amatrix}
\end{equation*}
The augmented matrix $(A|\vec{b})$ can be row-reduced into reduced row-echelon
form.
\footnotesize
\begin{align*}
	\begin{amatrix}{4}
		1 & 2 & 3 & 5 & 8 \\
		2 & 4 & 6 & 10 & 16 \\
		1 & 1 & 1 & 1 & 10
	\end{amatrix}
	&\xrightarrow{\frac{1}{2}r_2 \rightarrow r_2}
	\begin{amatrix}{4}
		1 & 2 & 3 & 5 & 8 \\
		1 & 2 & 3 & 5 & 8 \\
		1 & 1 & 1 & 1 & 10
	\end{amatrix}
	\xrightarrow{r_2 - r_1 \rightarrow r_2}
	\begin{amatrix}{4}
		1 & 2 & 3 & 5 & 8 \\
		0 & 0 & 0 & 0 & 0 \\
		1 & 1 & 1 & 1 & 10
	\end{amatrix}
	\\
	&\xrightarrow{r_2 \leftrightarrow r_3}
	\begin{amatrix}{4}
		1 & 2 & 3 & 5 & 8 \\
		1 & 1 & 1 & 1 & 10 \\
		0 & 0 & 0 & 0 & 0
	\end{amatrix}
	\xrightarrow{r_2 - r_1 \rightarrow r_2}
	\begin{amatrix}{4}
		1 & 2 & 3 & 5 & 8 \\
		0 & -1 & -2 & -4 & 2 \\
		0 & 0 & 0 & 0 & 0
	\end{amatrix}
	\\
	&\xrightarrow{-r_2 \rightarrow r_2}
	\begin{amatrix}{4}
		1 & 2 & 3 & 5 & 8 \\
		0 & 1 & 2 & 4 & -2 \\
		0 & 0 & 0 & 0 & 0
	\end{amatrix}
	\xrightarrow{r_1 - 2r_2 \rightarrow r_1}
	\begin{amatrix}{4}
		1 & 0 & -1 & -3 & 12 \\
		0 & 1 & 2 & 4 & -2 \\
		0 & 0 & 0 & 0 & 0
	\end{amatrix}
\end{align*}
\normalsize
\newpage
\noindent The reduced row echelon form of $(A|\vec{b})$,
\begin{equation*}
	\begin{amatrix}{4}
		1 & 0 & -1 & -3 & 12 \\
		0 & 1 & 2 & 4 & -2 \\
		0 & 0 & 0 & 0 & 0
	\end{amatrix}
	,
\end{equation*}
is analogous to the matrix equation
\begin{equation*}
	\begin{pmatrix}
		1 & 0 & -1 & -3 \\
		0 & 1 & 2 & 4 \\
		0 & 0 & 0 & 0
	\end{pmatrix}
	\begin{pmatrix} x \\ y \\ z \\ w \end{pmatrix}
	=
	\begin{pmatrix} 12 \\ -2 \\ 0 \end{pmatrix}
\end{equation*}
and also to the system
\begin{align*}
	x -z - 3w &= 12 \\
	y + 2z + 4w &= -2
	.
\end{align*}
In this system, $z$ and $w$ are free variables. \\
Thus, the system has \boxed{\text{infinitely many solutions}}.
\\[\baselineskip]
Solutions to the system are of the form
\begin{equation*}
	\boxed{
		\begin{pmatrix}
			12 + z + 3w \\
			-2 - 2z - 4w \\
			z \\
			w
		\end{pmatrix}
	}
	,
\end{equation*}
or, equivalently,
\begin{equation*}
	\boxed{
		\begin{pmatrix} 12 \\ -2 \\ 0 \\ 0 \end{pmatrix}
		+
		\begin{pmatrix} 1 \\ -2 \\ 1 \\ 0 \end{pmatrix} z
		+
		\begin{pmatrix} 3 \\ -4 \\ 0 \\ 1 \end{pmatrix} w
	}
	.
\end{equation*}
\newpage

\section*{2.}
\textit{Include an example of appropriate matrix as you justify your responses
to the following questions.}
\begin{enumerate}[label=(\alph*)]
	\item \textit{Suppose a linear system having six equations and three
	unknowns is consistent.
	Can you guarantee that the solution is unique?
	Can you guarantee that there are infinitely-many solutions?}
	\\[\baselineskip]
	Consider the row-reduced system:
	\begin{equation*}
		\begin{pmatrix}
			0 & 0 & 0 \\
			0 & 0 & 0 \\
			0 & 0 & 0 \\
			0 & 0 & 0 \\
			0 & 0 & 0 \\
			0 & 0 & 0
		\end{pmatrix}
		\begin{pmatrix} x \\ y \\ z \end{pmatrix}
		=
		\begin{pmatrix} 0 \\ 0 \\ 0 \\ 0 \\ 0 \\ 0 \end{pmatrix}
		.
	\end{equation*}
	\boxed{\text{A unique solution cannot be guaranteed.}}
	\\[\baselineskip]
	Consider the row-reduced system:
	\begin{equation*}
		\begin{pmatrix}
			1 & 0 & 0 \\
			0 & 1 & 0 \\
			0 & 0 & 1 \\
			0 & 0 & 0 \\
			0 & 0 & 0 \\
			0 & 0 & 0
		\end{pmatrix}
		\begin{pmatrix} x \\ y \\ z \end{pmatrix}
		=
		\begin{pmatrix} 0 \\ 0 \\ 0 \\ 0 \\ 0 \\ 0 \end{pmatrix}
		.
	\end{equation*}
	The only solution to such a system is
	\begin{equation*}
		\begin{pmatrix} x \\ y \\ z \end{pmatrix}
		=
		\begin{pmatrix} 0 \\ 0 \\ 0 \end{pmatrix}
	\end{equation*}
	Thus, \boxed{\text{infinitely-many solutions cannot be guaranteed}}.
	\newpage
	\item \textit{Suppose that a linear system having three equations and six
	unknowns is consistent.
	Can you guarantee that the solution is unique?
	Can you guarantee that there are infinitely-many solutions?}
	\\[\baselineskip]
	Consider the row-reduced system:
	\begin{equation*}
		\begin{pmatrix}
			0 & 0 & 0 & 0 & 0 & 0 \\
			0 & 0 & 0 & 0 & 0 & 0 \\
			0 & 0 & 0 & 0 & 0 & 0
		\end{pmatrix}
		\begin{pmatrix} a \\ b \\ c \\ d \\ e \\ f \end{pmatrix}
		=
		\begin{pmatrix} 0 \\ 0 \\ 0 \end{pmatrix}
		.
	\end{equation*}
	\boxed{\text{A unique solution cannot be guaranteed.}}
	\\[\baselineskip]
	For there to be a unique solution, there must be a pivot for each column
	of the coefficient matrix in the row-reduced augmented matrix. This is not
	possible when there are more columns than rows in the coefficient matrix.
	Thus, \boxed{\text{infinitely-many solutions are guaranteed}}.
	\item \textit{Suppose that a linear system is consistent and has a unique
	solution.
	What can you guarantee about the pivot positions in the augmented matrix?}
	\\[\baselineskip]
	For a row-reduced augmented matrix of a consistent linear system with a
	unique solution, \boxed{\text{a pivot must exist for each column of the
	coefficient matrix}}.
\end{enumerate}

\end{document}