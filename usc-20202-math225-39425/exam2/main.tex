% Preamble.
\documentclass[12pt]{article}
\usepackage[margin=1.25in]{geometry}
\usepackage[fleqn]{amsmath}
\usepackage{textcomp}
\usepackage{gensymb}
\usepackage{amsfonts}
\usepackage{enumitem}
%\usepackage{tikz}  % Include for figures.
%\usepackage{subfiles}  % Include for subfiles.

%% Title macros.
\newcommand{\EXAMNUM}{2}
\newcommand{\NAME}{D. Choi}
\newcommand{\DATE}{2020-07-08}

\title{\vspace{-2\baselineskip}MATH 225 - Exam \#\EXAMNUM}
\author{\NAME}
\date{\DATE}

%% Formatting options.
%\pagenumbering{gobble}  % Include for single-page document.
\allowdisplaybreaks

% Macros.
%% Norm.
\newcommand{\norm}[1]{\left\lVert#1\right\rVert}
%% Rotation and reflection matrices.
\newcommand{\rota}[1]{\text{Rot}(#1 \degree)}
\newcommand{\refl}[1]{\text{Ref}(#1 \degree)}
%% Augmented matrix.
% https://tex.stackexchange.com/a/2238
\newenvironment{amatrix}[1]{%
  \left(\begin{array}{@{}*{#1}{c}|c@{}}
}{%
  \end{array}\right)
}
%% Contextualized by input/output bases.
\newcommand{\based}[3]{{\{#1\}}_{#2}^{#3}}


% Document.
\begin{document}
\maketitle

\section*{1.}
\textit{For each each of the ordinary differential equations listed below,
answer the following:
\begin{itemize}
	\item What is its order?
	\item Is it linear or non-linear?
	\item If non-linear, why?
	\item If linear, is it constant coefficient? Is it homogenous?
\end{itemize}
}
\begin{enumerate}[label=(\alph*)]
	\item $x^{\prime\prime\prime} + x \cdot x^{\prime\prime} + t x^\prime
	+ x = t^2$.
	\\[\baselineskip]
	Third-order, non-linear. \\
	It is non-linear because of the term $x \cdot x^{\prime\prime}$.
	
	\item $x^{\prime\prime\prime} + t x^{\prime\prime} + x = \text{sin}(t).$
	\\[\baselineskip]
	Third-order, linear. \\
	It is not constant coefficient because of the term $t x^{\prime\prime}$.
	\\
	It is not homogenous because of the term $\text{sin}(t)$.
	
	\item $x^2 + t x^\prime + x = \text{sin}(t)$.
	\\[\baselineskip]
	First-order, non-linear. \\
	It is non-linear because of the term $x^2$.
	
	\item $2x^{\prime\prime} + 4 x^\prime = -x$.
	\\[\baselineskip]
	Second-order, linear. \\
	It is constant coefficient because all of the coefficients are constant. \\
	It is homogenous because there are no constant terms.
\end{enumerate}
\newpage

\section*{2.}
\begin{enumerate}[label=(\alph*)]
	\item \textit{Solve
	$x^{\prime\prime} + 3x^\prime + 2x = 0$ with the restrictions that
	$x(0) = 2$ and $x^\prime(0) = 1$.}\\[\baselineskip]
	The roots of the characteristic equation are
	\begin{equation*}
		r^2 + 3r + 2 = 0 \quad \Rightarrow \quad r_1 = -1, \quad r_2 = -2.
	\end{equation*}
	With $2$ distinct real roots, the general solution is
	\begin{equation*}
		x(t) = c_1 e^{-t} + c_2 e^{-2 t}.
	\end{equation*}
	Using the system of equations given by the initial conditions,
	\begin{alignat*}{2}
		x(0) &=& \quad 2 &= c_1 + c_2 \\
		x^\prime(0) &=& \quad 1 &= -c_1 - 2 c_2,
	\end{alignat*}
	the coefficients can be solved for as
	\begin{equation*}
		c_1 = 5, \quad
		c_2 = -3.
	\end{equation*}
	Thus,
	\begin{equation*}
		\boxed{
			x(t) = 5 e^{-t} - 3 e^{-2 t}
		}.
	\end{equation*}
	
	\item \textit{Consider a similar, but non-homogenous o.d.e.:
	$x^{\prime\prime} + 3x^\prime + 2x = 2t + 3$. Show that
	$x(t) = 7e^{-2t} + 2e^{-t} + t$ solves this non-homogenous o.d.e.}
	\\[\baselineskip]
	Let's ``brute forge plug and chug'':
	\begin{align*}
		x &= 7e^{-2t} + 2e^{-t} + t \\
		x^\prime &= -14e^{-2t} - 2e^{-t} + 1 \\
		x^{\prime\prime} &= 28e^{-2t} + 2e^{-t}
	\end{align*}
	Thus,
	\begin{align*}
		2t + 3 &\overset{?}{=} x^{\prime\prime} + 3x^\prime + 2x
		\\ &\overset{?}{=}
		(28e^{-2t} + 2e^{-t})
		+ 3(-14e^{-2t} - 2e^{-t} + 1)
		+ 2(7e^{-2t} + 2e^{-t} + t)
		\\ &\overset{?}{=}
		(28 - 42 + 14)(e^{-2t})
		+ (2 - 6 + 4)(e^{-t})
		+ (2)(t) + (3)(1)
		\\
		2t + 3 &= 2t + 3.
	\end{align*}
\end{enumerate}
\newpage

\section*{3.}
\textit{Suppose that}
\begin{equation*}
	A = 
	\begin{pmatrix}
		-1 & 1 & 1 \\
		1 & -1 & 1 \\
		1 & 1 & -1
	\end{pmatrix}
	\begin{pmatrix}
		1 & 0 & 0 \\
		0 & 2 & 0 \\
		0 & 0 & -1
	\end{pmatrix}
	\begin{pmatrix}
		-1 & 1 & 1 \\
		1 & -1 & 1 \\
		1 & 1 & -1
	\end{pmatrix}^{-1}
	.
\end{equation*}
\begin{enumerate}[label=(\alph*)]
	\item \textit{What is
		$A^{1001} \begin{pmatrix} 0 \\ 2 \\ 0 \end{pmatrix}$
	?}
	\\[\baselineskip]
	Note that
	\begin{equation*}
		\begin{pmatrix} 0 \\ 2 \\ 0 \end{pmatrix}
		=
		\begin{pmatrix} -1 \\ 1 \\ 1 \end{pmatrix}
		+
		\begin{pmatrix} 1 \\ 1 \\ -1 \end{pmatrix}
		.
	\end{equation*}
	Thus,
	\begin{align*}
		A^{1001} \begin{pmatrix} 0 \\ 2 \\ 0 \end{pmatrix}
		&=
		A^{1001} \left(
			\begin{pmatrix} -1 \\ 1 \\ 1 \end{pmatrix}
			+
			\begin{pmatrix} 1 \\ 1 \\ -1 \end{pmatrix}
		\right)
		\\
		&=
		A^{1001} \begin{pmatrix} -1 \\ 1 \\ 1 \end{pmatrix}
		+
		A^{1001} \begin{pmatrix} 1 \\ 1 \\ -1 \end{pmatrix}
		\\
		&=
		\lambda_{1}^{1001} \begin{pmatrix} -1 \\ 1 \\ 1 \end{pmatrix}
		+
		\lambda_{3}^{1001} \begin{pmatrix} 1 \\ 1 \\ -1 \end{pmatrix}
		\\
		&=
		1^{1001} \begin{pmatrix} -1 \\ 1 \\ 1 \end{pmatrix}
		+
		{(-1)}^{1001} \begin{pmatrix} 1 \\ 1 \\ -1 \end{pmatrix}
		\\
		&=
		\begin{pmatrix} -1 \\ 1 \\ 1 \end{pmatrix}
		-
		\begin{pmatrix} 1 \\ 1 \\ -1 \end{pmatrix}
		\\
		&=
		\boxed{
			\begin{pmatrix} -2 \\ 0 \\ 2 \end{pmatrix}
		}
	\end{align*}
	
	\item \textit{What is $\text{det}(A)$? Why?}
	\begin{align*}
		\text{det}(A)
		&= \text{det}(S) \text{det}(D) \text{det}(S^{-1}) \\
		&= \text{det}(S) \text{det}(D) \frac{1}{\text{det}(S)} \\
		&= \text{det}(D) \\
		&=
		\begin{vmatrix}
			1 & 0 & 0 \\
			0 & 2 & 0 \\
			0 & 0 & -1
		\end{vmatrix} \\
		&= 1 \cdot 2 \cdot -1 \\
		&= \boxed{-2}.
	\end{align*}
	
	\item \textit{What is an eigenbasis for $A$?}
	\begin{equation*}
		\boxed{
			\left\{
				\begin{pmatrix} -1 \\ 1 \\ 1 \end{pmatrix},
				\begin{pmatrix} 1 \\ -1 \\ 1 \end{pmatrix},
				\begin{pmatrix} 1 \\ 1 \\ -1 \end{pmatrix}
			\right\}.
		}
	\end{equation*}
\end{enumerate}
\newpage

\section*{4.}
\textit{Let $A$ be the $3 \times 3$ matrix associated with the transformation
that orthogonally projects vectors onto the plane $2x + 3y - z = 0$. \\
Let $B$ be the $3 \times 3$ matrix associated with the transformation that
reflects vectors across the plane $x + y + z = 0$.}
\begin{enumerate}[label=(\alph*)]
	\item \textit{Pick one of the following and find a basis for it:
	\begin{itemize}
		\item $\text{nullspace}(AB)$
		\item $\text{range}(AB)$
	\end{itemize}
	}
	The range of $AB$ is the plane that $A$ projects onto. \\
	A basis for the range of $AB$ can be found using the normal vector of the
	plane,
	\begin{equation*}
		\vec{n} = \begin{pmatrix} 2 \\ 3 \\ -1 \end{pmatrix},
	\end{equation*}
	to find two linearly independent vectors orthogonal to $\vec{n}$ which lie
	on the plane:
	\begin{equation*}
		\boxed{
			\left\{
				\begin{pmatrix} 1 \\ 0 \\ 2 \end{pmatrix},
				\begin{pmatrix} 0 \\ 1 \\ 3 \end{pmatrix}
			\right\}.
		}
	\end{equation*}
	
	\item \textit{Repeat the above instructions for:
	\begin{itemize}
		\item $\text{nullspace}(BA)$
		\item $\text{range}(BA)$
	\end{itemize}
	}
	The nullspace of $BA$ consists of all the vectors that $A$ will
	orthographically project to $0$. \\
	Thus, a basis for the nullspace of $BA$ is
	\begin{equation*}
		\left\{ \vec{n} \right\}
		=
		\boxed{
			\left\{
				\begin{pmatrix} 2 \\ 3 \\ -1 \end{pmatrix}
			\right\}.
		}
	\end{equation*}
	
	\item \textit{Find all of the following and briefly explain your
	reasoning.}
	\begin{itemize}
		\item $\text{rank}(AB) = 2$
		\item $\text{nullity}(AB) = 1$
		\item $\text{rank}(BA) = 2$
		\item $\text{nullity}(BA) 1$
	\end{itemize}
	The image of both $AB$ and $BA$ are planes, thus the rank of both is $2$. \\
	By the rank-nullity theorem, the nullities of $AB$ and $BA$ must be
	$3 - 2 = 1$.
\end{enumerate}
\newpage

\section*{5.}
\textit{Let $T$ be the reflection in $\mathbb{R}^2$ across $y=x$.}
\begin{gather*}
	\mathcal{A}
	=
	\left\{
		\begin{pmatrix} \cos(30 \degree) \\ \sin(30 \degree) \end{pmatrix},
		\begin{pmatrix} -\sin(30 \degree) \\ \cos(30 \degree) \end{pmatrix}
	\right\}
	\\
	\mathcal{B}
	=
	\left\{
		\begin{pmatrix} -\sin(45 \degree) \\ \cos(45 \degree) \end{pmatrix},
		\begin{pmatrix} \cos(45 \degree) \\ \sin(45 \degree) \end{pmatrix}
	\right\}
\end{gather*}
\textit{Find the following and show your work.}
\begin{enumerate}[label=(\alph*)]
	\item
	\begin{align*}
		\based{T}{\mathcal{B}}{\mathcal{A}}
		&=
		\based{T}{\mathcal{S}}{\mathcal{A}}
		\based{T}{\mathcal{S}}{\mathcal{S}}
		\based{T}{\mathcal{B}}{\mathcal{S}}
		\\ &=
		\begin{pmatrix}
			\cos(30 \degree) & -\sin(30 \degree) \\
			\sin(30 \degree) & \cos(30 \degree)
		\end{pmatrix}^{-1}
		\begin{pmatrix} 0 & 1 \\ 1 & 0 \end{pmatrix}
		\begin{pmatrix}
			-\sin(45 \degree) & \cos(45 \degree) \\
			\cos(45 \degree) & \sin(45 \degree
		\end{pmatrix}
	\end{align*}
	
	\item
	\begin{align*}
		\based{T}{\mathcal{A}}{\mathcal{A}}
		&=
		\based{T}{\mathcal{S}}{\mathcal{A}}
		\based{T}{\mathcal{S}}{\mathcal{S}}
		\based{T}{\mathcal{A}}{\mathcal{S}}
		\\ &=
		\begin{pmatrix}
			\cos(30 \degree) & -\sin(30 \degree) \\
			\sin(30 \degree) & \cos(30 \degree)
		\end{pmatrix}^{-1}
		\begin{pmatrix} 0 & 1 \\ 1 & 0 \end{pmatrix}
		\begin{pmatrix}
			\cos(30 \degree) & -\sin(30 \degree) \\
			\sin(30 \degree) & \cos(30 \degree)
		\end{pmatrix}
	\end{align*}
	
	\item
	\begin{align*}
		\based{T}{\mathcal{A}}{\mathcal{A}}
		&=
		\based{T}{\mathcal{S}}{\mathcal{A}}
		\based{T}{\mathcal{S}}{\mathcal{S}}
		\based{T}{\mathcal{A}}{\mathcal{S}}
		\\ &=
		\begin{pmatrix}
			-\sin(45 \degree) & \cos(45 \degree) \\
			\cos(45 \degree) & \sin(45 \degree
		\end{pmatrix}^{-1}
		\begin{pmatrix} 0 & 1 \\ 1 & 0 \end{pmatrix}
		\begin{pmatrix}
			-\sin(45 \degree) & \cos(45 \degree) \\
			\cos(45 \degree) & \sin(45 \degree
		\end{pmatrix}
	\end{align*}
\end{enumerate}
\newpage

\section*{6.}
\textit{Let $E$ be the $25 \times 25$ matrix with ones on both diagonals and
zeros everywhere else.}
\begin{enumerate}[label=(\alph*)]
	\item \textit{For each vector, determine if it is an eigenvector of $E$.
	If it is, find the corresponding eigenvalue. If it is not, give a brief
	explanation why.}
	{
	\itemize{
		\item \textit{$\vec{e}_1$ = 1 on top, 24 0s.}
		\\[\baselineskip]
		No, $E \vec{e}_1$ will have 1 on the top and bottom with zeros
		everywhere else.
		
		\item \textit{$\vec{e}_2$ = 12 0s, 1 in the middle, 12 0s.}
		\\[\baselineskip]
		Yes, $E \vec{e}_2 = \vec{e}_2$, thus $\boxed{\lambda_2 = 1}$.
		
		\item \textit{$\vec{e}_3$ = 1 on top, 23 0s, 1 on bottom.}
		\\[\baselineskip]
		Yes, $E \vec{e}_3$ will have 2 on top, 23 0s, and 2 on the bottom.
		Thus, $\boxed{\lambda_3 = 2}$.
		
		\item \textit{$\vec{e}_4$ = 0, 1 on top, 21 0s, 1, 0 on bottom.}
		\\[\baselineskip]
		Uh, yes, $\boxed{\lambda_4 = 2}$.
		
		\item \textit{$\vec{e}_5$ = 1 on top, 23 0s, -1 on bottom.}
		\\[\baselineskip]
		Yes, $E \vec{e}_5 = \vec{0}$, thus $\boxed{\lambda_5 = 0}$.
	}
	}
	
	\item \textit{Create a complete algebraic and geometric multiplicity table
	for the eigenvalues of $E$. Explain your reasoning.}
	Ouch.
	
	\item \textit{Is $E$ diagonalizable? Justify.} \\[\baselineskip]
	Yes. This was a guess.
	
	\item \textit{Is $E$ invertible? Justify.} \\[\baselineskip]
	No, row-reducing the matrix will result in the bottom half being zeros.
\end{enumerate}

\end{document}