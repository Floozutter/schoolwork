\documentclass[12pt]{article}
\usepackage[fleqn]{amsmath}
\usepackage{tikz}
\usepackage{subfiles}

\title{\vspace{-4\baselineskip}MATH 225 - Homework \#2}
\author{D. Choi}
\date{2020-05-22}

%\pagenumbering{gobble}

\begin{document}
\maketitle

\section*{1.}
\textit{Find the matrix that reflects space across the
$y = \tan(25^\circ)x$
line.
} \\[\baselineskip]
Let $A$ be such a matrix. \\
The elements of $A$ can be found by computing where the standard basis vectors
will land after being reflected across the
$y = \tan(25^\circ)x$
line.
\begin{equation*}
	A
	\begin{pmatrix}
		1 \\
		0
	\end{pmatrix}
	=
	A
	\begin{pmatrix}
		\cos0 \\
		\sin0
	\end{pmatrix}
	=
	\begin{pmatrix}
		\cos(2 \cdot 25^\circ) \\
		\sin(2 \cdot 25^\circ)
	\end{pmatrix}
	=
	\begin{pmatrix}
		\cos(50^\circ) \\
		\sin(50^\circ)
	\end{pmatrix}
\end{equation*}
\begin{equation*}
	A
	\begin{pmatrix}
		0 \\
		1
	\end{pmatrix}
	=
	A
	\begin{pmatrix}
		\cos(90^\circ) \\
		\sin(90^\circ)
	\end{pmatrix}
	=
	\begin{pmatrix}
		\cos(90^\circ - 2 \cdot 65^\circ) \\
		\sin(90^\circ - 2 \cdot 65^\circ)
	\end{pmatrix}
	=
	\begin{pmatrix}
		\cos(-40^\circ) \\
		\sin(-40^\circ)
	\end{pmatrix}
\end{equation*}
\begin{equation*}
	A =
	\boxed{
		\begin{pmatrix}
			\cos(50^\circ) & \cos(-40^\circ) \\
			\sin(50^\circ) & \sin(-40^\circ)
		\end{pmatrix}
	}
\end{equation*}

\section*{2.}
\begin{equation*}
	M =
	\begin{pmatrix}
		\frac{1}{\sqrt{2}} & \frac{1}{\sqrt{2}} \\
		-\frac{1}{\sqrt{2}} & \frac{1}{\sqrt{2}}
	\end{pmatrix}
\end{equation*}
\textit{Find $M^2$, $M^3$, $M^4$, $M^{17}$.} \\[\baselineskip]
Let $R(\theta)$ be the rotation matrix where
\begin{equation*}
	R(\theta) =
	\begin{pmatrix}
		\cos \theta & -\sin \theta \\
		\sin \theta & \cos \theta
	\end{pmatrix}
	.
\end{equation*}
$M$ can be considered as a specific instance of $R(\theta)$ where
$\theta = -45^\circ$.
\begin{equation*}
	M =
	\begin{pmatrix}
		\frac{1}{\sqrt{2}} & \frac{1}{\sqrt{2}} \\
		-\frac{1}{\sqrt{2}} & \frac{1}{\sqrt{2}}
	\end{pmatrix}
	=
	\begin{pmatrix}
		\cos(-45^\circ) & -\sin(-45^\circ) \\
		\sin(-45^\circ) & \cos(-45^\circ)
	\end{pmatrix}
	=
	R(-45^\circ)
\end{equation*}
Because $R(\theta)$ has the property of
\begin{equation*}
	(R(\theta))^n = R(\theta n),
\end{equation*}
it can be said for $M$ that
\begin{equation*}
	M^n = R(-45^\circ \cdot n).
\end{equation*}
Thus,
\begin{gather*}
	\boxed{M^2 = R(-45^\circ \cdot 2) = R(-90^\circ)}, \\
	\boxed{M^3 = R(-45^\circ \cdot 3) = R(-135^\circ)}, \\
	\boxed{M^4 = R(-45^\circ \cdot 4) = R(-180^\circ)},\text{ and} \\
	\boxed{M^{17} = R(-45^\circ \cdot 17) = R(-45^\circ) = M} .
\end{gather*}

\section*{3.}
\textit{Which is linear?}
\begin{gather*}
	f(x,y) = \langle 3x^2 + y, 2x + y \rangle \\
	g(x,y) = \langle 3x + y, x + 3y \rangle
\end{gather*}
\textit{Find the corresponding matrix for the linear transformation.}
\bigskip
\begin{equation*}
	f(2
	\begin{pmatrix}
		1 \\
		0
	\end{pmatrix}
	)
	=
	f
	\begin{pmatrix}
		2 \\
		0
	\end{pmatrix}
	=
	\begin{pmatrix}
		3(2)^2 + (0) \\
		2(2) + 0
	\end{pmatrix}
	=
	\begin{pmatrix}
		12 \\
		4
	\end{pmatrix}
\end{equation*}
\begin{equation*}
	2f
	\begin{pmatrix}
		1 \\
		0
	\end{pmatrix}
	=
	2
	\begin{pmatrix}
		3(1)^2 + 0 \\
		2(1) + 0
	\end{pmatrix}
	=
	2
	\begin{pmatrix}
		3 \\
		2
	\end{pmatrix}
	=
	\begin{pmatrix}
		6 \\
		4
	\end{pmatrix}
\end{equation*}
As
\begin{equation*}
	f(2
	\begin{pmatrix}
		1 \\
		0
	\end{pmatrix}
	)
	\neq
	2f
	\begin{pmatrix}
		1 \\
		0
	\end{pmatrix},
\end{equation*}
$f$ is not homogeneous of degree 1, a property which linearity requires. \\
Thus, $f$ is not linear.
\begin{equation*}
	g
	\begin{pmatrix}
		x \\
		y
	\end{pmatrix}
	=
	\begin{pmatrix}
		3x + y \\
		x + 3y
	\end{pmatrix}
	=
	\boxed{
		\begin{pmatrix}
			3 & 1 \\
			1 & 3
		\end{pmatrix}
		\begin{pmatrix}
			x \\
			y
		\end{pmatrix}
	}
\end{equation*}

\section*{4.}
\textit{Use geometry (graph paper) to find}
\begin{equation*}
	\begin{pmatrix}
		4 & 1 \\
		1 & 4
	\end{pmatrix}
	\begin{pmatrix}
		1 \\
		1
	\end{pmatrix}
	.
\end{equation*}
\bigskip
\subfile{fig}
\begin{equation*}
	\begin{pmatrix}
		4 & 1 \\
		1 & 4
	\end{pmatrix}
	\begin{pmatrix}
		1 \\
		1
	\end{pmatrix}
	=
	\boxed{
		\begin{pmatrix}
			5 \\
			5
		\end{pmatrix}
	}
\end{equation*}

\end{document}